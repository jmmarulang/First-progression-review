%% ----------------------------------------------------------------
%% LiteratureReview.tex
%% ---------------------------------------------------------------- 
\section{Literature Review} \label{section:LiteratureReview}

In this section we start by reviewing some approaches to QLs, as well as their quantification. Since our investigation tends their application as DLs, we focus only on their numerical interpretation (as opposed of, e.g. a possibilistic interpretation \citep{LIU19981}). %Table \mcita{} shows a summary of the numerical interpretations and properties of some relevant QLs. 
We then review how DLs have been studiend for and applied to \InAI{} systems. Lastly, we review how interactive theorem provers have been leveraged for AI verification.

\textbf{Fuzzy logics and substructural logics.} Fuzzy logics were introduced via the idea that truth is a matter of degree, and were some of the first logics to leave the Boolean interpretation \citep{galatos2007residuated}. The full exposition of their significance is beyond the scope of this review (for more information, see \cite{cintula2011handbook, prooffuzzy}). Broadly, they model the interval $[0,1]$ and their implication is defined as the residuum of conjunction, which in part is defined as a \emph{triangular norm} (t-norm) \citep{cintula2011handbook,prooffuzzy}. Fuzzy logics with left-continuous t-norms belong to the family of substructural logics, i.e. logics with Gentzen-style calculi that lack some structural rules from classical logic \citep{galatos2007residuated}. Linear logic is another example of a substructural logic \citep{Wadler1993, agliano2025algebraic, galatos2007residuated}. Some relevant fuzzy logics of this kind include Łukasiewicz, Product and G\"{o}del \citep{cintula2011handbook,prooffuzzy}. It is well known that residuated lattices, i.e. a lattices with a monoidal operator that meet the residuation law \citep{galatos2007residuated}, give complete algebraic semantics for propositional substructural logics \citep{galatos2007residuated}. However, it is not trivial how to extend these logics into first-order without loosing completeness. Under the standard interpretation of quantifiers as infimum and supremum \citep{rescher1969many, cintula2011handbook}, the first-order extension of Gödel logic is the only one, among the most prominent fuzzy logics, that is sound and complete w.r.t. models with values in $[0,1]$ \citep{cintula2011handbook}. Other approaches for fuzzy quantifiers have been studied, such as t-quantifier \citep{LIU19981}, generalized means \citep{badreddine2022logic, slusarz2023logic} and other aggregation operators \citep{LIU19981}. Yet it remains unclear to us how they affect the deductive systems and semantics. Recently, \citeauthor{slusarz2023logic} proposed interpreting quantifiers as expected values, and developed a sound first-order sequent calculus for several fuzzy logics \citep{slusarz2023logic}.

%\jnote{Not sure where to find more information on the proof theoretic study of different quantifier semantics for fuzzy logics.}

\textbf{Logics of the lawvere quantile.} As their name suggests, logics of the Lawvere quantile are those which model the Lawvere quantile of extended positive reals (that is, $[0,\infty]$)  \citep{bacci2023propositional}. This structure generalizes residuated lattices, that substructural logics model \citep{galatos2007residuated}. Their main motivation is providing foundations for quantitative reasoning. \citeauthor{bacci2023propositional} formalized a logic of this kind, and showed that the Łukasiewicz logic is also an instance of this quantile \citep{bacci2023propositional, bacci2024polynomial}. They provide a sound deductive system for their logic, and prove completeness for finitely axiomatizable theories. \citeauthor{bacci2025induction} further extended \citeauthor{bacci2024polynomial}'s logic with an induction principle, provided an algebraic interpretation, and showed how it can be applied to encode Markov Processes \citep{bacci2025induction}. There is currently no first-order extension for this logic. As mentioned in the introduction, \citeauthor{capucci2024quantifiers} also presented a logic that takes the Lawvere quantile as its interpretation, while also resembling linear logic \citep{Wadler1993, agliano2025algebraic}. Its monoidal operator is similar to that of product logic \cite{cintula2011handbook, prooffuzzy}, and possesses a family of parametrised operators that approximate lattice operators at the limit. \citeauthor{capucci2024quantifiers}'s logic is under development, yet it promises many relevant symbolic and subsymbolic properties.

\textbf{Logics from the \InAI{} community.} Prompted by the growing interest on \InAI{}, QLs have been developed with the specific purpose of being applied as DLs. However, most remain understudied from a proof-theoretic perspective. \emph{Probabilistic Circuits} have been used to embed knowledge during training in NNs \citep{lee2009advances, braun2025tractablerepresentationlearningprobabilistic}. \citeauthor{kimmig2012short} introduce \emph{Probabilistic Soft Logic} \citep{kimmig2012short}, that resembles  Łukasiewicz logic \cite{cintula2011handbook,prooffuzzy}. \citeauthor{fischer2019dl2} introduce the \emph{DL2} language \citep{fischer2019dl2}, that resembles product fuzzy logic \citep{cintula2011handbook, prooffuzzy}. \emph{Signal Temporal Logic} (STL) is a variant of temporal logic with a numerical interpretation \citep{varnai2020robustness}. Not unlike \citeauthor{capucci2024quantifiers}'s logic, STL has a softness parameter that turns its conjunction and disjunction into lattice operators at the limit \citep{varnai2020robustness}. The connectives of this logic, however, are not associative (except at the limit) \citep{affeldt2024taming}. It is worth noting that none of the previous logics possess a known deductive system. Several unified languages for DLs have also been proposed \citep{badreddine2022logic, van2024uller, slusarz2023logic}. \citeauthor{serafini2016logic} introduce \emph{Real Logic}, a first order language with groundings (what we refer to as interpretations) over $[0,1]$, and implement it in deep \emph{Logic Tensor Networks} (LTN) \cite{badreddine2022logic}.   
They define satisfiability in terms of confidence intervals, and introduced the concepts of \emph{satisfiability error} and \emph{approximate satisfiability}, useful for reasoning on LTNs.  \citeauthor{van2022analyzing} propose some groundings for Real logic \citep{van2022analyzing}. Explicit semantics for finitary quantifiers of Real Logic were later studied by \citeauthor{badreddine2022logic}, who proposed grounding them as the smooth maximum and minimum \citep{badreddine2022logic}. They also introduced a new grounding, nicked \emph{Stable Product Real logic} \citep{badreddine2022logic}, 
%\jnote{Do you think this stable product logic may be worth exploring from the proof theoretic side?}
a modified version of product logic \citep{van2022analyzing}. Similar to Real logic, \citeauthor{van2024uller} proposed the \emph{Uller} framework, which gives both a fuzzy and a probably interpretations \citep{van2022analyzing}. \citeauthor{schellhorn2025muller} recently provided categorical semantics for this framework, as well as numerical semantics for first-order Product Real logic \citep{schellhorn2025muller}. \citeauthor{slusarz2023logic} differenciate from previous work by providing a well-typed and sound calculus for their framework, interpreting quantifiers as expected values, nicked \emph{Logic of Differentiable Logics},  \citep{slusarz2023logic}. 

\textbf{Subsymbolic study of DLs.} From the subsymbolic side, \citeauthor{van2022analyzing} analyse some gradient properties of fuzzy logic operators, and mention some common problems: \emph{single-passing}, \emph{vanishing gradients}, and \emph{exploding gradients} \cite{van2022analyzing}. They conclude that product logic with Reichenbach implication is the best performing. \citeauthor{badreddine2022logic} extend this analysis to some aggregate operators and propose Stable Product Real Logic as an alternative with better gradient properties \citep{badreddine2022logic}. \citeauthor{varnai2020robustness} propose \emph{geometric properties} desirable for DLs \citep{varnai2020robustness}:  \emph{scale-invariance}, \emph{weak smoothness} and \emph{shadow-lifting} (\cref{Shadow-lifting}), among others;  and study STL \citep{varnai2020robustness} from this lense. However, they also prove that a DL cannot meet shadow-lifting while being idempotent and associative \citep{varnai2020robustness}. Building on this, \citeauthor{affeldt2024taming} analyse wherever some prominent DLs  meet the shadow-lifting property \citep{affeldt2024taming, varnai2020robustness}. We speculate that \citeauthor{capucci2024quantifiers} is able to balance shadow-lifitng and idempotence by adding additional operators that approximate lattice operations \citep{capucci2024quantifiers}. \citeauthor{FLINKOW2025103280} perform an empirical analysis of some DLs as loss functions, and make use of formal verification tools to evaluate their ability to provide guarantees \citep{FLINKOW2025103280}. 

\textbf{AI formalization.} We build on the work of \citeauthor{affeldt2024taming}, who formalize some prominent propositional DLs in Rocq \citep{affeldt2024taming} using the syntax of the logic of differentiable logics \cite{slusarz2023logic}. Machine-learning backends have been previously mechanized in Agda \citep{agdaDL}, as part of the formalization of the Vehicle language \citep{vehicle}.  Property-guided training certified via mechanization was also proposed by \citeauthor{chevallier2022constrainedtrainingneuralnetworks} \citep{chevallier2022constrainedtrainingneuralnetworks}. Relevant formalizations of \SuAI{} related concepts include: verification of NNs in Isabelle/HOL \citep{brucker2023verifying} and Imandra \citep{desmartin2022checkinn}, formalisation of piecewise affine activation functions in Rocq \citep{aleksandrov2023formalizing}, providing generalization guarantees in Rocq \citep{bagnall2019certifying}, convergence of a single-layered perceptron in Rocq \citep{murphy2017verified}; and verification of neural archetypes in Rocq \citep{DeMaria2021}. The formalisation presented here does not directly formalise neural networks. On the other hand, many \SiAI{} concepts have been mechanized. Some relevant examples include a formalization of BI in Rocq \citep{10.1145/3497775.3503690}, and a formalization of the deep inference MAV logic in Agda \citep{Atkey2024}. 
