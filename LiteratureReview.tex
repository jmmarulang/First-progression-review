%% ----------------------------------------------------------------
%% LiteratureReview.tex
%% ---------------------------------------------------------------- 
\section{Literature Review} \label{section:LiteratureReview}

In this section we start by reviewing some approaches to QLs, as well as their quantification. Since our investigation cares about their application as DLs, we focus only on their numerical interpretation (as opposed of a, e.g. possibilistic interpretation \mcita{}). Table \mcita{} shows a summary of their numerical interpretations and properties. We then review how DLs have been applied to \InAI{} systems and their impact. Lastly, we review how interactive theorem provers have been leveraged for AI verification.

\textbf{Fuzzy logics and substructural logics.} Fuzzy logics were introduced via the idea that truth is a matter of degree, and were some of the first logics to leave the Boolean interpretation \mcita{}. The full exposition of their significance is beyond the scope of this review (for more information, see \mcita{}). Generally, they model the interval $[0,1]$ and their implication is defined as the residuum of conjunction, which in part is defined as a \emph{triangular norm} (t-norm) \mcita{}. Fuzzy logics with left-continuous t-norms belong to the family of substructural logics, i.e. logics with deductive systems that lack some structural rules from classical Gentzen calculi \mcita{}. Some relevant fuzzy logics of this kind include ŁukasiewiczIt, Product and G\"{o}del \mcita{}. It is well known that residuated lattices give complete algebraic semantics for propositional substructural logics \mcita{}. However, it is not trivial how to extended these logics into first-order without loosing completeness. Under the standard interpretation of quantifiers as infimum and supremum \mcita{}, the first-order extension of Gödel logic is the only one, among the most prominent fuzzy logics \mcita{}, that is sound and complete w.r.t. models with values in $[0,1]$ \mcita{}. While other approaches for fuzzy quantifiers have been studied (such as t-quantifier and other aggregation operators \mcita{}) it remains unclear how they affect the deductive systems.

\textbf{Logics of the lawvere quantile.}

\textbf{Logics from the \InAI{} community.}

\textbf{Applications in \InAI{}.}

\textbf{Applications in \InAI{}.}

\TODO
