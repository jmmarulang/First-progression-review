%% ----------------------------------------------------------------
%% LiteratureReview.tex
%% ---------------------------------------------------------------- 
\chapter{Literature Review} \label{Chapter:LiteratureReview}

%Broadly speaking, the proof-theoretic study of \InAI{} can be related to the more general effort of reconciling 
\jnote{I add the subsection to give an idea of how the text could be structured. They can be removed later on}

\jnote{Is there any work that deals with the proof-theoretic study of some kind of \InAI{} ?}

\section{Mathematical Foundations of Quantitative Logics}
\subsection{Fuzzy logics}
\yada. QLs have been studied for decades, and date back to the ideas of Kleene, G\"{o}del, and Łukasiewicz at the start of the 20th century \citep{cintula2011handbook, prooffuzzy}. Among them, the oldest and most understood are \emph{Fuzzy Logics} \citep{cintula2011handbook, prooffuzzy}. These are logics where conjunction is given by a \emph{Triangular Norm} \cite{cintula2011handbook, prooffuzzy} 
\jnote{And other norms?}
and relate to the family of \emph{Substructural Logics} \citep{galatos2007residuated}. Some relevant fuzzy logics include \yada. Fuzzy logics have been extended into first-order with the infimum and supremum as models for the universal and existencial quantifiers \mcita{}. However, these extensions are not trivial and come with their own quirks \yada.
Fuzzy logics appear in \AI{} as \yada
\jnote{Where to find information on first-order extensions of fuzzy logics?}

\subsection{Logics of the Lawvere Quantile}

Logics from the \emph{Lawvere Quantile} \mcita{} have been proposed to reconcile logic and quantitative reasoning \yada. Recently Capucci proposed \yada 

Logics of the lawvere quantile appear in \AI{} as \yada

\subsection{Differentiable Logics}
\jnote{There will be some overlap here. In this subsection I want to focus on logics proposed by the machine learning community}
\yada

From the \SiAI{} perspective, some desirable properties of DLs include \yada
From the \SuAI{} perspective, some desirable properties of DLs include \yada
\section{Applications of QLs in \InAI{}}
Beyond providing possible semantics for \InAI{}, and therefore theoretical guarantees for backends \yada, QLs have also been applied more directly \yada

\subsection{As property driven training/loss functions}
\yada there is also an interest in providing programming language support for DLs \yada Vehicle and Caisar \yada
\subsection{As Logic Tensor Networks/Fuzzy circuits}

%While the proof-theoretic study of \SiAI{} systems is an old and well established field \mcita{}, the more prominent examples of \SuAI{}, such as \NN{}s, remain obscure. Therefore, a possible approximation 




%There are several approaches to the proof-theoretic study of systems. 

%QLs have proven to be a promising approach for giving semantics to \InAI{} \mcita{}. 




\TODO
