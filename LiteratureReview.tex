%% ----------------------------------------------------------------
%% LiteratureReview.tex
%% ---------------------------------------------------------------- 
\section{Literature Review} \label{section:LiteratureReview}

In this section we start by reviewing some approaches to QLs, as well as their quantification. Since our investigation tends their application as DLs, we focus only on their numerical interpretation (as opposed of, e.g. a possibilistic interpretation \citep{LIU19981}). %Table \mcita{} shows a summary of the numerical interpretations and properties of some relevant QLs. 
We then review how DLs have been studiend for and applied to \InAI{} systems. Lastly, we review how interactive theorem provers have been leveraged for AI verification.

\textbf{Fuzzy logics and substructural logics.} Fuzzy logics were introduced via the idea that truth is a matter of degree, and were some of the first logics to leave the Boolean interpretation \citep{galatos2007residuated}. The full exposition of their significance is beyond the scope of this review (for more information, see \cite{cintula2011handbook, prooffuzzy}). Generally, they model the interval $[0,1]$ and their implication is defined as the residuum of conjunction, which in part is defined as a \emph{triangular norm} (t-norm) \citep{cintula2011handbook,prooffuzzy}. Fuzzy logics with left-continuous t-norms belong to the family of substructural logics, i.e. logics with Gentzen-style calculi that lack some structural rules from classical logic \citep{galatos2007residuated}. Linear logic is another example of a substructural logics \citep{agliano2025algebraic, galatos2007residuated}. Some relevant fuzzy logics of this kind include Łukasiewicz, Product and G\"{o}del \citep{cintula2011handbook,prooffuzzy}. It is well known that residuated lattices, i.e. a lattice with a monoidal operator that meets the residuation law \citep{galatos2007residuated}, give complete algebraic semantics for propositional substructural logics \citep{galatos2007residuated}. However, it is not trivial how to extended these logics into first-order without loosing completeness. Under the standard interpretation of quantifiers as infimum and supremum \citep{rescher1969many}, the first-order extension of Gödel logic is the only one, among the most prominent fuzzy logics, that is sound and complete w.r.t. models with values in $[0,1]$ \citep{cintula2011handbook}. \citeauthor{slusarz2023logic} propose interpreting quantifiers as expected values, and develop a sound first-order sequent calculus for several fuzzy logics \citep{slusarz2023logic}. Other approaches for fuzzy quantifiers have been studied, such as t-quantifier \citep{LIU19981}, generalized means \citep{badreddine2022logic, slusarz2023logic} and other aggregation operators \citep{LIU19981}. Yet it remains unclear to us how they affect the deductive systems and semantics.

\jnote{Not sure where to find more information on the proof theoretic study of different quantifier semantics fro fuzzy logics.}

\textbf{Logics of the lawvere quantile.} As their name suggests, logics of the Lawvere quantile are those which model the Lawvere quantile of extended positive reals (that is, $[0,\infty]$)  \citep{bacci2023propositional}. This structure generalizes residuated lattices, that substructural logics model \citep{galatos2007residuated}. Their main motivation is providing foundations for quantitative reasoning. \citeauthor{bacci2023propositional} formalized a logic of this kind, and showed that the Łukasiewicz logic is also an instance of this quantile \citep{bacci2023propositional, bacci2024polynomial}. They provide a sound deductive system for their logic, and prove completeness for finitely axiomatizable theories. \citeauthor{bacci2025induction} further extended \citeauthor{bacci2024polynomial}'s logic with an induction principle, provided an algebraic interpretation, and showed how it can be applied to encode Markov Processes \citep{bacci2025induction}. There is currently no first-order extension for this logic. As mentioned in the introduction, \citeauthor{capucci2024quantifiers} also presented a logic that takes the Lawvere quantile as its interpretation, while also resembling linear logic \citep{agliano2025algebraic}. Its monoidal operator is similar to that of product logic \cite{cintula2011handbook, prooffuzzy}, and possess a family of parametrised operators that approximate lattice operators at the limit. \citeauthor{capucci2024quantifiers}'s logic is under development, yet it promises many relevant symbolic and subsymbolic properties.

\textbf{Logics from the \InAI{} community.}

\textbf{Subsymbolic study of DLs.} From the subsymbolic side, \citeauthor{van2022analyzing} analyse some gradient properties of fuzzy logic operators \cite{van2022analyzing}. They conclude that product logic with Reichenbach implication is the best performing. \citeauthor{badreddine2022logic} extend this analysis to some aggregate operators and propose a modified version of product logic with better gradient properties \citep{badreddine2022logic}. \citeauthor{varnai2020robustness} propose \emph{geometric properties} desirable for DLs \citep{varnai2020robustness}, and study STL \mcita{} from this lense. Building on this, \citeauthor{affeldt2024taming} analyse wherever some prominent DLs  meet the shadow-lifting property \citep{affeldt2024taming, varnai2020robustness}. We speculate that \citeauthor{capucci2024quantifiers} is able to balance shadow-lifitng and idempotence by adding additional operators that approximate lattice operations \citep{capucci2024quantifiers}. \citeauthor{FLINKOW2025103280} perform an experimental heavy analysis of some DLs as loss functions, and make use of formal verification tools to evaluate their ability to provide guarantees \citep{FLINKOW2025103280}. 

\textbf{Theorem provers for AI.}

\TODO
