%% ----------------------------------------------------------------
%% Thesis.tex
%% ---------------------------------------------------------------- 
\documentclass[sotoncolour]{uosthesis}      % Use the Thesis Style with custom link colour
\graphicspath{{Figures/}}   % Location of your graphics files
\usepackage[round]{natbib}            % Use Natbib style for the refs.
\setcitestyle{numbers}
\usepackage{bibentry}          % Use bibentry for prepublished works
\nobibliography*               % Use bibentry for prepublished works
\usepackage{attrib}            % Use the attrib package for quotations
\hypersetup{colorlinks=true}   % Set to false for black/white printing
\input{Definitions}            % Include your abbreviations

%%My packages

%%comments
\usepackage{comment}
\usepackage{todonotes}

%%References
\usepackage[nameinlink]{cleveref}
\crefname{Definition}{Definition}{Definitions}
\crefname{Lemma}{Lemma}{Lemmas}
\setcitestyle{square}

%%Math
\usepackage{stmaryrd}
\usepackage{amsmath}
\usepackage{syntax}
\usepackage{empheq}
\usepackage{proof}
\usepackage{bussproofs}
\usepackage{paralist} 

%graphics
\usepackage{graphicx}
\usepackage{float}
\usepackage{pdfpages}

%code
\usepackage{minted}
\usemintedstyle{tango}

 
%%Alias 
\defcitealias{Platzer_2024}{Intersymbolic AI}

%% ----------------------------------------------------------------
%% --------------------THESIS/DOC INFORMATION ---------------------
\department  {School of Electronics and Computer Science}
\DEPARTMENT  {\MakeUppercase{\deptname}}
\group       {Cyber Physical Systems}
\GROUP       {\MakeUppercase{\groupname}}
\faculty     {Faculty of  Engineering and Physical Sciences}
\FACULTY     {\MakeUppercase{\facname}}
\title      {First Progression Review}
%% TODO: Replace with your name removing []
\authors    {Jairo Miguel Marulanda-Giraldo} % Use of Soton Email unadvised, use ORCiD instead.
\addresses  {\groupname\\\deptname\\\univname}
\date       {\today}
%% Optional Fields TODO: Replace these fields with your own data

\qualifications{Bs Applied Mathematics}
%\orcidid{0000-0002-1825-0097}
%\doi{10.1002/0470841559.ch1}
%\volume{n}{m} %Optional Volume Numbering Volume n of m
\subject    {}
\keywords   {}
\supervisor{Ekaterina Komendantskaya\\Enrico Manchioni\\Alessandro Bruni}

\begin{document}
%% ------------------ FRONT MATTER ORGANISATION -------------------
\pagenumbering{gobble} % removes page number
%\copyrightDeclaration{} % !!! Comment this line when printing the hardcopy !!!
%\raggedright                  %% Set the style to Left justification, remove for fill justification
                              %% Must be done after copyrightDeclaration
%\frontmatter
\maketitle
\begin{abstract}

Quantitative logics (QLs) model subsets of the real numbers, and have increasingly been applied for neurosymbolic learning. In particular, their real semantics can be used to integrate verification properties during learning. At the same time, QLs have been applied to give programming language support for property-driven training. Yet, current QLs either cannot guarantee the correctness of their compilers, or hinder the performance of the resulting artifacts. In this project, we develop and study \textit{\OurLogic{}}, a logic that is both well-behaved and appropriate for training, using the Mathematical Components library in the Rocq proof assistant. We aim to develop a sound and complete first-order calculus with respect to algebraic semantics. Our approach takes inspiration from both fuzzy logics and the logic of bunched implications.

\end{abstract}

%\tableofcontents
%\listoffigures
%\listoftables
%% The List of listings does not, by default, appear in the ToC, so....
%\addtotoc{Listings}
%\lstlistoflistings
%\listofaddmaterial
%\addtolom{Material Name e.g Map}
%\addtolom{Material Name e.g CD}
%\addtolom{Test Material}
%% ---------- AUTHORSHIP DECLARATION/ ACKNOW. / DEDICATORY ----------
%% Either include citations like below (as many as required spaced with commas or 'and').
%% \bibentry command must be used here with prepublished papers
%\authorshipdeclaration{\bibentry{Gunn:2001:pdflatex}\newline\bibentry{Lovell:2011:updated}\newline\bibentry{Gunn:2011:updated2}}
%% Or state no citations like below
%% \authorshipdeclaration{}
%% -----------------------
%\acknowledgements{Thanks to no one.}
%\dedicatory{To \dots}
%%Lightweight Definitions and Abbreviations see package:nomencl for alternative
%% Include if relevant to discipline
%\listofsymbols{ll}{$w$ & The weight vector\\$\S$ & If relevant to discipline}
%\mainmatter
%% ------------------ MAIN MATTER (CONTENT) --------------------

%% ----------------------------------------------------------------
%% Introduction.tex
%% ---------------------------------------------------------------- 
\section{Problem Statement} \label{section:Problem Statement}

\textbf{The two flavours of Artificial Intelligence.}
Broadly speaking, there exists  two very distinct approaches to \emph{\AILong{}}  (\emph{\AI{}}): One rooted in reasoning and another in learning \citep{Platzer_2024, booch2021thinking}.  \SiAI{} (also referred to as good old fashioned AI \citep{haugeland1989artificial}, classical AI \citep{garnelo2019reconciling},  or logic-based AI \citep{thomason2003logic}) relates to algebraic computing, and emphasizes finding analytical solutions by manipulating logical expressions. This approach, by principle, prioritizes interpretability and preserving meaning. Current relevant examples include \emph{SMT/SAT solvers} \citep{barrett2018satisfiability, alyahya2022structure}, \emph{theorem provers} \citep{bartek2025vampire, barras1999coq},  as well as many \emph{programming languages} (PL) \citep{korner2022fifty, perkel2019julia, klabnik2023rust}.
%\jnote{Not sure about what to reference for programming languages} 
\SiAI{} has had a broad impact on planning \citep{geffner2013concise}, gameplay \citep{newell1958chess} and system verification \citep{leroy2016compcert, tihanyi2025new}, as well as on the field of mathematics \cite{Blokpoel2024}.  On the other hand, \SuAI{} (also referred to as pattern engines \citep{julia2020there})
%\jnote{I like the term "universal approximators", based on the universal approximation theorem that states that a neural network can approximate any continuous function to any desired degree of accuracy. I feel it better portrais what subsymbolic AI is, and would like to add it to the list of alternative names. But as far as I know its not been introduced before.}
relates to numerical computing,  and focuses on approximating solutions by applying statistical and optimization methods. They are often data driven, and do not require an explicit algorithm to operate (beyond the indirect computations performed to approximate the result). The more relevant examples of \SuAI{} are \emph{Deep Learning} \citep{norvig2002modern} and \emph{Reinforcement Learning} \citep{sutton1998reinforcement}, but older methods such as \emph{Kalman Filters} \citep{simon2001kalman} or \emph{Monte Carlo Simulation} \citep{martin2024computing} could arguably also fall under this category.  \SuAI{} has recently grown in quality and proliferated to many applications such as image/language processing \citep{thapa2024application, vaswani2017attention}, and simulation \citep{jumper2021highly}. 

\textbf{\InAI{}.} In order to leverage their respective strengths, there is a growing interest in studying and developing methods that merge Symbolic and Subsymbolic AI. This broad category, nicked \emph{\InAI{}}  \citep{Platzer_2024} (also referred to as neuro-symbolic AI \citep{d2009neural}, or hybrid intelligent systems \citep{medsker2012hybrid}), can range from applying logical principles on the architecture of \emph{ \NN{}s} (NN)  \citep{badreddine2022logic, petersen2022deep}, to using \SuAI{} to generate better heuristics for theorem provers \citep{laurent2022learning}. On top of this, \InAI{} has found particular success in cyber-physical systems, where it often plays the role of a controller \citep{Platzer_2024}.

\textbf{\DL{}.} One of the main challenges of \InAI{} is the grounding of symbolic knowledge into numerical representations. A promising approach is the study of \emph{\QL{}s} (QL), i.e.~logics that model the real numbers. They have been studied for decades, and date back to the ideas of Kleene, G\"{o}del, and Łukasiewicz at the start of the 20th century \citep{cintula2011handbook, prooffuzzy}. Some relevant QLs include \emph{Fuzzy Logics} \citep{cintula2011handbook} and the \emph{Logics of The Lawvere Quantile} \citep{bacci2024polynomial, bacci2023propositional, capucci2024quantifiers, bacci2025induction}. To illustrate QLs, let us have a toy syntax with atomic propositions and conjunction, such as
\begin{equation}
\begin{split}
    \Phi \ni \phi &:= A \,|\, \phi \land \phi
\end{split}
\end{equation}
where $\phi$ is interpreted through a mapping $\tempty{\cdot} : \Phi \rightarrow D$ such that $ \tempty{\phi} \in D \subseteq \Ereal$. $D$ varies among logics and restricts the interpretation of connectives. For example, the
G\"{o}del logic \citep{BAAZ200723} has a standard semantics over $[0, 1]$ where the conjunction is interpreted as the minimum function. \emph{\DL{}s} (DL) form a family of methods that apply key insights from QLs to \InAI{}. DLs have been applied on property-driven training \citep{FLINKOW2025103280}, safe-by-construction systems \citep{badreddine2022logic}, and SAT solving \citep{kyrillidis2021continuous, gaglione2022maxsat}. 

\textbf{What makes a good DL.} To function as a bridge between symbolic and subsymbolic AI, a DL would benefit from certain properties. From the \SiAI{} perspective, a DL should be expressive enough to encode properties of interest (e.g. see \citep{vehicle}). Even more, to certify its implementation, a DL should possess a deductive system, as well as some form of soundness and completeness proofs with respect to a semantics \citep{floyd1993assigning, goguen1977initial}. From the \SuAI{} perspective, differentiability of its interpretation is an obvious candidate; continuity or convexity are also widely considered desirable; \citeauthor{varnai2020robustness} also suggest characterizing DLs in terms of their \textit{geometric properties} \citep{varnai2020robustness}. Out of these, the most notable is \emph{Shadow-lifting}:

\begin{definition}[Shadow-lifting]
    \label[Definition]{Shadow-lifting}
    A logical operator $A : \Phi ^ n \rightarrow \Phi$ satisfies the shadow-lifting if, for any $\tempty{\phi} \neq 0$ and $i \in [1,n]$,
    \begin{equation*}
	\left. \dfrac{\partial \tempty{A(\phi_1, ..., \phi_n)}}{\partial \tempty{\phi_i}_L}\right\rvert_{\phi_j = \phi \text{ where } i \neq j} >0
	\end{equation*}
	where $ \partial $ denotes partial differentiation.
\end{definition}

\jnote{Is this equivalent to saying that the gradient of $A$ is always positive component-wise? It is NOT but its strangely similar. The only difference is that in shadow-lifting, the partial derivative must only be positive if every other component has been replaced by the SAME constant. Which seems an oddly specific condition to me. I feel this definition could be generalized.}

However, it has been shown that a an operator cannot meet shadow-lifting while being associative and idempotent \citep{varnai2020robustness}. \citeauthor{van2022analyzing} also mention three gradient problems to be avoided that commonly arise from logic operators: \emph{single-passing}, \emph{vanishing}, and \emph{exploding gradients} \citep{van2022analyzing}. Moreover, it is not well understood the effects that DLs have on performance \citep{flinkow2025generalised}, as well as on the statistical and probabilistic guarantees of certain systems, such as generalization error boundaries \citep{jakubovitz2019generalization}.

%\jnote{Mention shadow lifting doesnt allow assoc or idemp}

\textbf{The quantifier problem.} Nevertheless, there is one fundamental problem that DLs face: Many properties of interest for machine learning involve quantifiers, yet the majority of QLs are propositional \citep{bacci2024polynomial, bacci2023propositional, bacci2025induction}. A canonical specification of this kind is  \textit{robustness} \citep{casadio2022neural},  i.e. small perturbations to the inputs of a neural network should result in small changes to its output, formally:
\begin{definition}[$\epsilon$-$\delta$-Robustness] % no space here
\label{Robustness}%
    Let $\epsilon, \delta \in \real^+$, $||\cdot||$ be a norm, and $f : \real^n \rightarrow \real^m$ be a measurable function.
    One says \textit{$f$ is $\epsilon$-$\delta$-robust} around $\bar x \in \real ^ n$ if 
    \begin{equation}
    \label{eq:robustness}
        \forall x\in \real^n , ||x - \bar x|| \leq \epsilon \Rightarrow || 
			f(x) - f(\bar x)|| \leq \delta  
    \end{equation}
\end{definition}

Expanding some sound and complete propositional QLs to first-order logic often comes at the expense of either completeness or continuity \citep{cintula2011handbook, slusarz2023logic}.  
%For example, the first-order extension of Gödel logic is the only one, among the most prominent fuzzy logics \mcita{}, that is sound and complete w.r.t. models with values in $[0,1]$ and with universal and existential quantifiers interpreted as infima and suprema \mcita{}.However, connectives of this logic are not continuous and therefore not suitable for gradient-descent algorithms.

\textbf{A novel approach.} Given the provided information, can we develop a first-order DL without loosing any of the desirable properties?   Recently, a promising approach for first-order QLs was proposed by \citeauthor{capucci2024quantifiers}: interpreting quantifiers as \textit{generalized means}, while introducing a "softness" modality that balances shadow-lifting and idempotence \citep{capucci2024quantifiers}. We build on these ideas to develop a novel well-behaved DL. Unlike \citeauthor{capucci2024quantifiers}, who uses a deep-inference inspired framework \citep{guglielmi2007system, guglielmi2015deep}, our approach follows the subtructural logic tradition \citep{galatos2007residuated}, taking elements from both bunched logic \citep{o1999logic} (an extension of linear logic \citep{Wadler1993, agliano2025algebraic}) and fuzzy logics \citep{cintula2011handbook, prooffuzzy}. We leverage this to give algebraic semantics for our DL, while studying its relation to other substructural logics and \citeauthor{capucci2024quantifiers}'s logic. 
To provide assurance of the correctness of our results, we aim to mechanise our proofs in Rocq, making use of its Mathematical Components library (\mathcomp{}) \cite{mathcomp}. We seek this formalization to be a stepping stone for the development of programming language support for verification of \InAI{} \citep{vehicle}. 

\subsection{Research Requirements}
The preceding introduction highlights two groups of properties desirable for DLs: \emph{Symbolic} and \emph{Subsymbolic} properties. 

\textbf{Symbolic properties.} For our logic to be expressive, it should possess negation, conjunction and disjunction operators. Negation should be \emph{involutive}; conjunction and disjunction should be \emph{commutative}, \emph{associative} and \emph{idempotent}; and all operators should \emph{distribute} in the classical manner \citep{galatos2007residuated}. If a logic meets all previous properties we say it is \emph{compositional}. Similarly, we would like our universal quantifiers to be \emph{aggregation operators} \citep{LIU19981}, and existencials to be their duals \citep{LIU19981}. If a logic meets these properties, we say it is \emph{aggregative}.  We also aim for our DL to be \emph{sound} and \emph{complete}. Intuitively, a logic is sound if any sentence that is provable in its deductive system is also true on its semantics \citep{galatos2007residuated}. Conversely, a logic is complete if any sentence that is true on its semantics is also provable in its deductive system \citep{galatos2007residuated}. While soundness provides theoretical guarantees of correctness, completeness is needed to guarantee that the syntax and the semantics of the logic match.

\textbf{Subsymbolic properties.} The interpretation of our logic should be \emph{differentiable}, 
\jnote{Should we focus on almost everywhere differentiable functions instead? It is more general and the minimum and maximum fall under this category (infimum does not). Its what they use to analyse DL2 in \cite{fischer2019dl2}.}
and meet the following geometric properties proposed by \citeauthor{varnai2020robustness}: \emph{scale-invariance}, \emph{weak smoothness} and \emph{shadow-lifting} \citep{varnai2020robustness}. \citeauthor{van2022analyzing} mention three gradient problems to be avoided that commonly arise from logic operators: \emph{single-passing}, \emph{vanishing gradients}, and \emph{exploding gradients} \citep{van2022analyzing}. All previous properties are useful for optimization.  

\subsection{Research Questions}

Our high-level goal is to develop and study a DL that meets the symbolic and subsymbolic properties. To this aim, we divide our research into six research questions:

\begin{enumerate}
    \item \textbf{Sound propositional DL.} Can we develop a sound, compositional, and propositional DL, while maintaining subsymbolic properties? 
    
    Our current DL could be summarized as a "soft" bunched logic \citep{o1999logic} with hypersequents from fuzzy logic \citep{prooffuzzy}. Like \citeauthor{capucci2024quantifiers}, our DL makes use of a modality to balance shadow-lifting and compositionality. This modality is also applied to bunches. Our DL has been proven sound with respect to its numerical interpretation. We are currently researching how to develop algebraic semantics.
    
    \item \textbf{Complete propositional DL.} Can we prove this propositional DL complete?
    
    A first approximation to completeness is to prove that a collection of axioms that are true in the semantics, are provable (e.g. prelinearity \citep{prooffuzzy}). While this does not guarantee completeness, it is an informative necessary condition. Then, we hope to leverage our DLs relation to substructural logics to prove completeness by applying known techniques from fuzzy logic \citep{cintula2011handbook, galatos2007residuated}. It remains unclear if this is possible. 

    \jnote{Include formalization intent somewhere in here? How to state it as a research question?}
    
    \item \textbf{Relation to other logics.} How does our logic relate to other logics? 
    
    Being an extension of linear logic \citep{Wadler1993, agliano2025algebraic}, our logic (or at least a subset of our logic) belongs to the family of substructural logics \citep{galatos2007residuated}, with its monoidal operator resembling that of product logic \citep{cintula2011handbook, prooffuzzy}. It remains unclear where in the substructural family it would be positioned. As for \citeauthor{capucci2024quantifiers}'s logic, they very much resemble each other, and possess the same numerical interpretation. However, each logic possesses a different definition of validity, and therefore of soundness and completeness \citep{galatos2007residuated, capucci2024quantifiers}. We expect this to affect the algebraic semantics as well, beyond just changing its defining equation \citep{galatos2007residuated, agliano2025algebraic}. 
    
    \item \textbf{Sound predicate DL.} Can our DL be extended into a sound aggregative logic, while maintaining subsymbolic properties?

    Extending into first-order requires modifying our numerical interpretation, and therefore our semantics. For our current approach to maintain soundness, additional conditions must be imposed on substitution. On the other hand, it remains unclear if subsymbolic properties are maintained. 
    
    \item \textbf{Complete predicate DL.} Can we prove our predicate DL complete?
    
    Similar approach to the propositional case. 
    
    \item \textbf{Practicality.} Does our propositional/predicate DL offer some insight on property-driven training?

    We would like to study the effects of our DL on the performance and generalisability \citep{jakubovitz2019generalization} of its implementations. This requires both theoretical and experimental evidence. 
    
\end{enumerate}


%% ----------------------------------------------------------------
%% LiteratureReview.tex
%% ---------------------------------------------------------------- 
\section{Literature Review} \label{section:LiteratureReview}

In this section we start by reviewing some approaches to QLs, as well as their quantification. Since our investigation cares about their application as DLs, we focus only on their numerical interpretation (as opposed of a, e.g. possibilistic interpretation \mcita{}). Table \mcita{} shows a summary of their numerical interpretations and properties. We then review how DLs have been applied to \InAI{} systems and their impact. Lastly, we review how interactive theorem provers have been leveraged for AI verification.

\textbf{Fuzzy logics and substructural logics.} Fuzzy logics were introduced via the idea that truth is a matter of degree, and were some of the first logics to leave the Boolean interpretation \mcita{}. The full exposition of their significance is beyond the scope of this review (for more information, see \mcita{}). Generally, they model the interval $[0,1]$ and their implication is defined as the residuum of conjunction, which in part is defined as a \emph{triangular norm} (t-norm) \mcita{}. Fuzzy logics with left-continuous t-norms belong to the family of substructural logics, i.e. logics with deductive systems that lack some structural rules from classical Gentzen calculi \mcita{}. Some relevant fuzzy logics of this kind include ŁukasiewiczIt, Product and G\"{o}del \mcita{}. It is well known that residuated lattices give complete algebraic semantics for propositional substructural logics \mcita{}. However, it is not trivial how to extended these logics into first-order without loosing completeness. Under the standard interpretation of quantifiers as infimum and supremum \mcita{}, the first-order extension of Gödel logic is the only one, among the most prominent fuzzy logics \mcita{}, that is sound and complete w.r.t. models with values in $[0,1]$ \mcita{}. While other approaches for fuzzy quantifiers have been studied (such as t-quantifier and other aggregation operators \mcita{}) it remains unclear how they affect the deductive systems.

\textbf{Logics of the lawvere quantile.}

\textbf{Logics from the \InAI{} community.}

\textbf{Applications in \InAI{}.}

\textbf{Applications in \InAI{}.}

\TODO


%% ----------------------------------------------------------------
%% CurrentApproach.tex
%% ---------------------------------------------------------------- 
\section{Background and Preliminary Results} \label{section:CurrentApproach}

In this section we introduce relevant background theory and summarize some preliminary results. We introduce a propositional version of \OL{} by its syntax, semantics, and sequent calculus. We finish we a proof of soundness with respect to the numerical interpretation of the logic.%We then mention our progress on extending it into first order and  on formalization.

\subsection{Preliminaries and Notation}
\label{Preliminaries}
%\subsubsection{Logic of the Reals}
%\subsubsection{\citeauthor{capucci2024quantifiers}'s logic}
We introduce preliminaries from the extended arithmetic of the reals. They are a modified version of \cite{capucci2024quantifiers}. We diverge from \emph{ibid.} in notation. Our base setting are the positive extended reals $[0,\infty]$. %, considered as sup-lattice with the usual order $\leq$.
%The topology on $\real^+$ is extended to $[0,\infty]$ by adding to the opens all the intervals $(a, \infty]$.
%As a measure space, $[0,\infty]$ is considered equipped with completion of its Borel $\sigma$-field (i.e. the Lebesgue $\sigma$-field); and then further equipped with the obvious extension of the Lebesgue measure given by setting $\lambda((a,\infty]) = \infty$ for $a < \infty$ and $\lambda(\{\infty\})=0$.

\begin{definition}[$p$-Sum]
\label{$p$-Sum}
    %On $[0,\infty]$, 
    \emph{$p$-sum} and \emph{harmonic $p$-sum} are, respectively, the following operations:
    \begin{equation*}
		\begin{tabular}{c|ccc}
			$a \psum{p} b$ & $0$ & $a \in (0,\infty)$ & $\infty$\\
			\cline{1-4}
			$0$ 			   & $0$ & $a$ 		& $\infty$\\
			$b \in (0,\infty)$ & $b$ & $(a^{p}+b^{p})^{1/p}$		& $\infty$\\
			$\infty$ 		   & $\infty$ & $\infty$ & $\infty$
		\end{tabular}
		\hspace*{10ex}
		\begin{tabular}{c|ccc}
			\textnormal{$a \phsum{p} b$} & $0$ & $a \in (0,\infty)$ & $\infty$\\
			\cline{1-4}
			$0$ 		 	   & $0$ 		& $0$ 	   & $0$\\
			$b \in (0,\infty)$ & $0$ 		& $(a^{-p}+b^{-p})^{-1/p}$	   & $b$\\
			$0$ 		   & $0$ 	& $a$ & $\infty$
		\end{tabular}
	\end{equation*}
    Where $p \in (0,\infty].$
\end{definition}

\begin{lemma}[Additive Collapse]
    \label[lemma]{AdditiveCollapse}
    The following are the limits of $\psum{p}$ and $\phsum{p}$ for $p \longrightarrow \infty$:
    \begin{equation*}
		\begin{tabular}{c|ccc}
			$a \psum{\infty} b$ & $0$ & $a \in (0,\infty)$ & $\infty$\\
			\cline{1-4}
			$0$ 			   & $0$ & $a$ 		& $\infty$\\
			$b \in (0,\infty)$ & $b$ & $\max{}(a,b)$		& $\infty$\\
			$\infty$ 		   & $\infty$ & $\infty$ & $\infty$
		\end{tabular}
		\hspace*{10ex}
		\begin{tabular}{c|ccc}
			\textnormal{$a \phsum{\infty} b$} & $0$ & $a \in (0,\infty)$ & $\infty$\\
			\cline{1-4}
			$0$ 		 	   & $0$ 		& $0$ 	   & $0$\\
			$b \in (0,\infty)$ & $0$ 		& $\min{}(a,b)$	   & $b$\\
			$\infty$ 		   & $0$ 	& $a$ & $\infty$
		\end{tabular}
	\end{equation*}
\end{lemma}

\begin{definition}[Multiplication]
\label{Multiplication}
    On $[0,\infty]$, \emph{conjunctive multiplication} and \emph{disjunctive multiplication} are, respectively, the following operations:
    \begin{equation*}
		\begin{tabular}{c|ccc}
			$a \conmul{} b$ & $0$ & $a \in (0,\infty)$ & $\infty$\\
			\cline{1-4}
			$0$ 			   & $0$ & $0$ 		& $0$\\
			$b \in (0,\infty)$ & $0$ & $ab$		& $\infty$\\
			$\infty$ 		   & $0$ & $\infty$ & $\infty$
		\end{tabular}
		\hspace*{10ex}
		\begin{tabular}{c|ccc}
			\textnormal{$a \dismul{} b$} & $0$ & $a \in (0,\infty)$ & $\infty$\\
			\cline{1-4}
			$0$ 		 	   & $0$ 		& $0$ 	   & $\infty$\\
			$b \in (0,\infty)$ & $0$ 		& $ab$	   & $\infty$\\
			$\infty$ 		   & $\infty$ 	& $\infty$ & $\infty$
		\end{tabular}
	\end{equation*}
\end{definition}

Notice $\conmul{}$ and $\dismul{}$ differ only when $a$ is $0$ and $b$ is $\infty$, or \textit{vice versa}. Often we write $ab$ instead of $a \conmul{} b$.

\begin{definition}[Division]
\label{Division}
    On $[0,\infty]$, \emph{division} is:
    \begin{equation*}
		\begin{tabular}{c|ccc}
			$a \ediv{} b$ & $0$ & $a \in (0,\infty)$ & $\infty$\\
			\cline{1-4}
			$0$ 			   & $\infty$ & $0$ 		& $0$\\
			$b \in (0,\infty)$ & $\infty$ & $b/a$		& $0$\\
			$\infty$ 		   & $\infty$ & $\infty$ & $\infty$
		\end{tabular}
	\end{equation*}
\end{definition}

\begin{definition}[Duality Operator]
\label{dual}
    Let $a \in [0,\infty]$. Then the \emph{dual} of $a$ is
    \[  \edual{a} =
    \begin{cases}
    1/a  & a \in (0,\infty)  \\
    \infty & a = 0 \\
    0 & a = \infty \\
   \end{cases}
    \]
\end{definition}

 Notice $a^{-1} = a \ediv{} 1 = a^{-1} \dismul{} b$, $a \dismul{} b = (a^{-1} \conmul{} b^{-1})^{-1}$, and $a \psum{p} b = (a^{-1} \phsum{p} b^{-1})^{-1}$. We sometimes slightly abuse the notation to write $a/b$ instead of $b \ediv a$ and $1/a$ instead of $\edual{a}$. 

Therefore,

\begin{lemma}
    \label[lemma]{IsLattice}
    $([0,\infty], \phsum{\infty}, \psum{\infty}, \conmul{}, \ediv, 1, 0, \infty)$ is a bounded commutative \emph{residuated lattice}:
    \begin{enumerate}
        \item $\phsum{\infty}$, $\psum{\infty}$ are commutative, associative and mutually absorptive, with units $\infty$ and $0$ respectively.
        \item $\conmul{}$ is associative and commutative, with unit element $1$.
        \item $a \conmul{} b \leq c$ iff $a \leq b \ediv{} c$.
    \end{enumerate}
\end{lemma}

\begin{lemma}[Some Properties]    \,
    \label[lemma]{SomePropertiesOfP}
    \begin{enumerate}
        \item \textbf{Additive Distributivity.} $(a \phsum{p} (b \psum{p} c)) \leq  ((a \phsum{p} b) \psum{p} (a \phsum{p} c))$.
        \item \textbf{Monotonicity.} If $a \leq b$ then $a \psum{p} c \leq b \psum{p} c$ and $a \phsum{p} c \leq b \phsum{p} c$ 
        \item \textbf{Semi-additivity.} $a \phsum{p} b \leq a \leq a \psum{p} b$,
        \item \textbf{Colax-distributivity.} $(a \phsum{p}  (b \psum{p} c) \leq (a \phsum{p}  b) \psum{p} (a \phsum{p}  c)$ .
        \item \textbf{Sub-distributivity.} $(a \dismul{}  b) \phsum{p} (c \dismul{}  d) \leq (a \dismul{}  c) \psum{p} (b \dismul{}  d)  $ .
        \\\\
        For $p \leq q$,
        \item \textbf{Conjunctive p-Monotonicity.} $a \phsum{p} b \leq  a \phsum{q} b$.
        \item \textbf{Disjunctive p-Monotonicity.} $a \psum{p} b \leq  a \psum{q} b$.
        \item \textbf{Disjunctive Duoidality.} $(a \psum{q} b) \psum{p} (c \psum{q}  d) \leq (a \psum{p}  c) \psum{q} (b \psum{p}  d)  $,
        \item \textbf{Conjunctive Duoidality.} $(a \phsum{p}  b) \phsum{q} (c \phsum{p}  d) \leq (a \phsum{q}  c) \phsum{p} (b \phsum{q}  d)  $ .
    \end{enumerate}
\end{lemma}

%\jnote{Should I include proofs?}

\begin{comment}
\subsection{Measure Spaces and $p$-means}\label{p-mean}

%\begin{definition}[Measurable Spaces and Functions]
%    Let $S, T$ be sets and $\sigmal{S}, \sigmal{T}$ be 
%    $\sigma$-algebras. The pairs $(S, \sigmal{S})$ and $(T, \sigmal{T})$
%     are \emph{measurable spaces}.\\ The function $f : S \rightarrow T$ is
%      \emph{measurable} over $S$ with values in $T$ if and only if for every 
%      $E \in \sigmal{T}$ the preimage of $E$ under $f$ is in $\sigmal{S}$, 
%      that is, for all $E \in 
%      \sigmal{T}$
%    \begin{equation}
%       f^{-1}(E) = \{x \in S_{1} \, | \,f(x) \in E \} \in \sigmal{S}.
%    \end{equation}
%\end{definition}

\begin{definition}[Measure Space]
    Let $S$ be a set and $\sigmal{S}$ be a $\sigma$-algebra over $S$. 
    A \emph{measure} on $(S,\sigmal{S})$ is a function 
    $\mu : \sigmal{S} \rightarrow [0,\infty]$ such that (1) $\mu (\varnothing) 
    = 0$ and (2) if $\{ A_i : i \in I \}$ is a countable collection of pairwise 
    disjoint sets in $\sigmal{S}$ then
    \begin{equation}
        \mu \left( \bigcup_{i \in I} A_i \right) = \sum_{i \in I} \mu (A_i).
    \end{equation}
    The triple $(S, \sigmal{S}, \mu)$ is called a \emph{measure space}, and a 
    \emph{probability space} when $\mu(S)=1$, in which case $\mu$ is often 
    denoted as $\mathbb{P}$.
\end{definition}

%\begin{definition}[Random Variable]
%    Let $S,T$ be measurable spaces. A \emph{random variable} over $S$ 
%    with values in $T$ is a measurable function 
%    $X : S \rightarrow T$.
%\end{definition}

We give the following definitions for positive functions only, since this is 
the integrals we use below.

\begin{definition}[Simple Functions and Lebesgue Integral]
    Let $(S, \sigmal{S})$ be a measurable space, $I$ be a finite index set, 
    $a_i \in \mathbb{R}$ for each $i \in I$ and $\{ A_i : i \in I\}$ a 
    collection of sets in $\sigmal{S}$. A \emph{simple function} on $S$ is one
     that can be written as a finite linear combination of indicator functions 
     of measurable subsets of $S$, i.e. one of the form 
     $f = \sum_{i \in I}a_{i}\emph{1}_{A_i}$.
    If $\emph{S} = (S, \sigmal{S}, \mu)$ is a measure space then:
    \begin{enumerate}
        \item If $f = \sum_{i \in I}a_{i}\emph{1}_{A_i}$ is a nonnegative 
        simple function, the \emph{Lebesgue integral} of $f$ is
        \begin{equation}
            \int_{\emph{S}} f = \int_{S} f(s) \, \mu(\de s) = 
            \sum_{i \in I} a_i \conmul \mu(A_i).
        \end{equation}
        \item If $f : S \rightarrow [0, \infty]$ is a measurable function, the \emph{Lebesgue integral} of $f$ is
        \begin{equation}
            \int_{\emph{S}} f = \int_{S} f(s) \, \mu(\de s) = 
            \sup\left\{ \int_{S} g : g \text{ is simple and } g \leq f \right\}.
        \end{equation}
    \end{enumerate}
\end{definition}

The following definitions relate specifically to the new quantifier semantics. 
They are what are classically known as generalized weighted means 
\cite{mitrinovic1970analytic}, though the geometric mean, much like multiplication 
above, bifurcates into a conjunctive and a disjunctive version.
\\\\
Throughout the following, fix a probability space 
$\emph{S} = (S, \sigmal{S}, \mathbb{P})$.


\begin{definition}[$p$-Means]
\label{pmean}
    Let $f : S \rightarrow \PEreal$ be a measurable function. For $p \in (0, \infty)$, the \emph{(generalized weighted) $p$-mean} of $f$ is
    \begin{equation}
        %\LMS{f}{p}{S} := \left(\int_{S} f(s)^p\, \mathbb{P}(\de s)(s)\right)^{1/p}
        \LMS{f}{p}{\emph{S}} := \left(\int_{\emph{S}} f^{\,p}\right)^{1/p} = 
        \left(\int_{S} f(s)^{\,p}\, \mathbb{P}(\de s)\right)^{1/p}
    \end{equation}
    where we extended the functions $(-)^p$ as follows
    \begin{equation}
        \infty^{p} =
        \begin{cases}
            1  & p = 0  \\
            \infty & p > 0
        \end{cases}
        \hspace*{10ex}
        0^{p} = 0.
    \end{equation}
    Dually, the \emph{(generalized weighted) harmonic $p$-mean} of $f$ is
    \begin{equation}
        \LMS{f}{-p}{\emph{S}} := 
        \left(\LMS{f^{-1}}{p}{\emph{S}}\right)^{-1} = 
        \left(\int_{\emph{S}} f^{\,-p}\right)^{-1/p} =
        \left(\int_{S} f(s)^{\,-p}\, \mathbb{P}(\de s)\right)^{-1/p}.
    \end{equation}
\end{definition}

When $\emph{S}$ can be inferred from the context (for example, when $f$ is a random 
variable), we write $\LM{f}{p}$.

The definition of $p$-means can be extended to $p=0$ and $p=\infty$ by taking limits \cite{capucci}. First we have

\begin{lemma}
\label{limitinfty}
    As $p \longrightarrow +\infty$,
    \begin{equation}
        \LM{f}{+p} \longrightarrow \esup{f} =: \LM{f}{+\infty},
        \qquad
        \LM{f}{-p} \longrightarrow \einf{f} =: \LM{f}{-\infty}.
    \end{equation}
\end{lemma}

These quantities are so defined:

\begin{definition}[Essential Extrema]
    Let $(S, \sigmal{S}, \mu)$ be a measure space and $f : S \rightarrow \PEreal$ a measurable function.
    \begin{enumerate}
        \item Let $U = \left\{ a \in \PEreal : \mu(\{ x \in X : a < f(x)\}) = 0\right\}$ and $\inf(U)$ be the infimum of U. The \emph{essential supremum} of $f$
        is
        \begin{equation}
            \esup{f} = \inf U
        \end{equation}
        recalling that $\inf \varnothing = \infty$.
        \item The \emph{essential infimum} of $f$ is
        \begin{equation}
            \einf{f} = - \,\esup{- f}
        \end{equation}
    \end{enumerate}
\end{definition}

On the other end of the spectrum, we have:

\begin{lemma}
\label{limitzero}
    As $p \longrightarrow 0$, both $\LM{f}{+p}$ and $\LM{f}{-p}$ converge to a limit, thus defining \emph{disjunctive} and \emph{conjunctive geometric means}:
    \begin{equation}
        \LM{f}{+p} \longrightarrow: \LM{f}{+0},
        \qquad
        \LM{f}{-p} \longrightarrow: \LM{f}{-0}.
    \end{equation}
\end{lemma}

For bounded functions, these quantities coincide with the classical (weighted) geometric mean:

\begin{definition}[Geometric Mean]
    Let $f : S \rightarrow [0,\infty)$ be a measurable function and $(S,\sigmal{S}, \mathbb{P})$ a measure space. The \emph{geometric mean} of $f$ is
    \begin{equation}
        GM[f] = \exp \left(\int_{S} \ln f(s)\, \mathbb{P}(\de s)\right)
    \end{equation}
\end{definition}

For unbounded functions, conjunctive and disjunctive geometric means may differ in the same way as $\conmul{}$ and $\dismul{}$, namely in the way they handle $0$ and $\infty$. See \cite{capucci2024quantifiers} for clarifications.
\end{comment}

\subsection{Syntax and Semantics}

\textbf{Syntax.}  \citeauthor{slusarz2023logic} propose a common syntax for all DLs \citep{slusarz2023logic}. Here we adapt a subset of it.  For simplicity we use the same symbols of \cref{Preliminaries} for our formulae. \cref{fig:syntax} defines the syntax of \OL{}. Types are given by Boolean or  Extended Positive Real Numbers ($(0,\infty]$). Formulae are freely generated from atomic propositions over logical connectives:\\

\begin{figure}[H]
\begin{subfigure}[t]{0.4\textwidth}
	\begin{grammar}
		<type> ::=  
        \BoolType | \ERealType 

        
	\end{grammar}
\end{subfigure}
\hfill
\begin{subfigure}[t]{1\textwidth}
	\begin{grammar}
		<exprEPReal> $p$ ::=  
        $p \in (0, \infty]$ 
	\end{grammar}
\end{subfigure}
\begin{subfigure}[t]{1\textwidth}
	\begin{grammar}
		<exprBool> $\ni \phi_{0},\phi_{1}$ ::=  
        $\unit$ | $\top$ | $\bot$ | $\phi_{0} \ediv{} \phi_{1}$ | $\phi_{0} \conmul{} \phi_{1}$ | $\phi_{0} \psum{\elEReal} \phi_{1}$ | $\phi_{0} \phsum{\elEReal} \phi_{1}$ 
	\end{grammar}
\end{subfigure}
\hfill
\setlength{\belowcaptionskip}{-20pt} 
	\caption{Types and expressions of \OL{}.\\
	}
	\label{fig:syntax}
\end{figure}
We can also encode 

$$\ldual{\phi} := \phi \ediv{} \unit \quad \phi_1 \dismul{} \phi_2 := \ldual{(\ldual{\phi_1} \conmul{} \ldual{\phi_2})}$$

\OL{}'s language resembles that of linear logic \citep{Wadler1993, agliano2025algebraic}, excluding that the additive connectives are parametrised by a positive extended real $p$. The intuition behind $p$ is that it regulates the degree of softness: the smaller $p$ is, the softer the claim.
The \emph{Multiplicative} operators are: The monoidal operator \emph{Tensor} ($\conmul{}$), the residual operator of tensor \emph{Linear Implication} ($\ediv$), the dual of tensor \emph{Par} ($\dismul{}$), and the unit of both tensor and par \emph{Unit} ($\unit$). The \emph{Soft Additive} operators are: The smallest element \emph{Bottom} ($\bot$), the biggest element \emph{Top} ($\top$), the generalized conjunction \emph{Soft Conjunction} ($\phsum{p}$), and the generalized disjunction \emph{Soft Disjunction} ($\psum{p}$). When $p = \infty$, the soft additives turn into the regular additives \citep{galatos2007residuated}.

\subsection{Sequent Calculus}

\textbf{Bunches.}
Bunches of formulae are defined following closely BI \cite{o1999logic}. Bunches are built using two operations, $(,)$ and $(;^p)$:
$$\Gamma, \Gamma_1, \Gamma_2 ::= \phi \ | \ 
\emptyset_{\times} \ | \ \emptyset_{+} \ 
|  \ \Gamma_1, \Gamma_2 \ | \ \Gamma_1;^p \Gamma_2 $$
Both are associative, commutative and unital w.r.t. their empty 
bunches, moreover, the former distributes over the latter.  These properties are summarised in Figure~\ref{fig:eqbunches}.  We write $\Gamma_{1}(-)$ for one-hole bunches. Plugging $\Gamma_{1}(\Gamma_{2})$ replaces the hole in $\Gamma_{1}$ with $\Gamma_{2}$.

 \begin{figure}[H]
	%\footnotesize
	\begin{spreadlines}{7pt}
		\begin{empheq}{gather*}
			(\Gamma_1, \Gamma_2), \Gamma_3  \Leftrightarrow \Gamma_1, 
			( \Gamma_2, \Gamma_3) \quad 
			\emptyset_{\times}, \Gamma  \Leftrightarrow \Gamma \quad
			\Gamma_1, \Gamma_2   \Leftrightarrow \Gamma_2, \Gamma_2
			\\
			(\Gamma_1;^p \Gamma_2) ;^p\Gamma_3  \Leftrightarrow \Gamma_1;^p 
			( \Gamma_2;^p \Gamma_3) \quad
           \emptyset_{+};^p \Gamma   \Leftrightarrow \Gamma \quad
           \Gamma_1;^p \Gamma_2   \Leftrightarrow \Gamma_2;^p \Gamma_1
			\\
			\Gamma_1, (\Gamma_2;^p \Gamma_3)  \Leftrightarrow  
			( \Gamma_1, \Gamma_2);^p (\Gamma_1,\Gamma_3)
			\\
			\AxiomC{$\Gamma_1 \Leftrightarrow \Gamma_2$}
			\RightLabel{\LJAxiom{}}
			\UnaryInfC{$\Gamma(\Gamma_1)  \Leftrightarrow  \Gamma(\Gamma_2)$}
			\bottomAlignProof
			\DisplayProof
		\end{empheq}
	\end{spreadlines}
	\vspace*{-1em}
	\caption{\footnotesize{Equivalence of Bunches.}}
	\label{fig:eqbunches}
\end{figure}

The idea of having the two operators is inspired by the fact that ($,$) 
corresponds to multiplicative connectives, and admits only one structural rule 
-- the exchange. On the other hand, ($;^{p}$) corresponds to the soft additive connectives, and in 
addition admits weakening and contraction (when $p = \infty$). The structural rules are given in Figure~\ref{fig:seq-rules}. 

\textbf{Hypersequents.}
Proofs will be given in terms of \emph{Hypsequents}, 
following the fuzzy logic tradition \citep{prooffuzzy}: 

$$\Hyp_1, \Hyp_2 ::= \sequentPDL{\AssumsEnv_1}{ \AssumsEnv_2} \ | \
(\Hyp_1 | \Hyp_2)
$$

Where $\AssumsEnv_1$ and $\AssumsEnv_2$ are bunches, $\sequentPDL{\AssumsEnv_1}{ \AssumsEnv_2}$ is a \emph{Sequent} and write $ \eHyp{} := \sequentPDL{\emptyset{}_{\times}}{\emptyset{}_{\times}}$. Hypersequents (as opposed to just sequents) will be required to prove prelinearity and distributivity, known to be valid formulae in \OL{} (\cref{SomeValidFormulae}), and therefore necessary for completeness. 

\textbf{Inference.} Proofs are defined inductively through the rules in \cref{fig:seq-rules}. The rules of \OL{} resemble those of Fuzzy Hypersequent Calculi \citep{prooffuzzy} and BI \citep{o1999logic}.  A few rules do not resemble either of these logics, and are needed in order to operate with soft bunches.  Following the order of Figure~\ref{fig:seq-rules}, the only initial hypersequent is $\eHyp{}$. The \emph{basic structural rules} resemble those of BI, with one notable caveat-- although weakening is always allowed, contraction of ($;^{p}$) is only permitted when $p = \infty$. In words, soft additives only turn intro true additives when softness is lost. 

The \emph{structural rules for Hypersequents} follow the ideas already present in the hypersequent fuzzy sequent calculi \citep{prooffuzzy}. The main motivation for introducing hypersequents is to ensure that the "prelinearity property" (as stated in \cref{SomeValidFormulae}) -- i.e. the property that reflect the total order of the real line --  is provable. Without the hypersequents, the property is only provable in the presence of weakening and contraction. Like fuzzy logics, we do not have the latter in full generality. Hypersequents facilitate the introduction of the \emph{communication} rules $\comM$ and $\comA$, that in turn facilitate the proof of pre-linearity for multiplicatives and additives, respectively. Note that $\comA$ does not exchange the consequents, to preserve soundness. 

The next two blocks of rules, for additive and multiplicative connectives, follows very closely the BI tradition \citep{o1999logic, 10.1145/3497775.3503690}, with the only deviation of using soft versions of additive conjunction and disjunction, $\phsum{p}$ and $\psum{p}$, respectively. It deserves to be mentioned that allowing bunches in the consequent is necessary in order to obtain a sound rule for $\orL$. Additionally, the condition $[\Gamma] \leq p$ in the rule for  $\orL$ is to be read as: the bunch $\Gamma$ does not contain soft modalities greater than $p$.

Lastly, $\pR$ and $\pL$ allow the manipulation of softness. These two rules, together with the restrictions on $\orL$ and $\IC$, mean that claims can only grow softer as we move down the proof.  

\begin{figure}[H]
	\footnotesize{
		\begin{spreadlines}{7pt}
			\begin{empheq}{gather*}
			\def\ScoreOverhang{1pt}
			\def\defaultHypSeparation{\hskip .15in}
			\def\labelSpacing{2pt}
			\def\ScoreOverhang{1pt}
			\def\labelSpacing{2pt}
			\textrm{\bf Initial hypersequents:}
			\\
			\eHyp
			\\
			\textrm{\bf Basic structural rules:}
			\\
			\AxiomC{$\sequentPDL{\AssumsEnv'}{\phi} \quad 
				\AssumsEnv \Leftrightarrow \AssumsEnv'$}
			%\AxiomC{$\AssumsEnv \Leftrightarrow \AssumsEnv'$}
    		\RightLabel{\Equi}
			\UnaryInfC{$\sequentPDL{\AssumsEnv}{\phi}$}
			\bottomAlignProof
			\DisplayProof
			\quad
			\AxiomC{$\eHyp$}
    		\RightLabel{\Ass}
			\UnaryInfC{$\sequentPDL{\phi}{\phi}$}
			\bottomAlignProof
			\DisplayProof
        	\\
			\AxiomC{$\sequentPDL{\Gamma(\Gamma_1)}{\AssumsEnv'}$}
    		\RightLabel{\IW-L}
			\UnaryInfC{$\sequentPDL{\Gamma(\Gamma_1;^p \Gamma_2)}{\AssumsEnv'}$}
			\bottomAlignProof
			\DisplayProof
			\quad
			\AxiomC{$\sequentPDL{\Gamma(\Gamma_1;^{\infty} \Gamma_1)}{\AssumsEnv'}$}
    		\RightLabel{\IC-L}
			\UnaryInfC{$\sequentPDL{\Gamma(\Gamma_1)}{\AssumsEnv'}$}
			\bottomAlignProof
			\DisplayProof
			\\
			\AxiomC{$\sequentPDL{\Gamma}{\AssumsEnv'(\AssumsEnv_1)}$}
    		\RightLabel{\IW-R}
			\UnaryInfC{$\sequentPDL{\Gamma}{\AssumsEnv'(\AssumsEnv_1;^{p} \AssumsEnv_2)}$}
			\bottomAlignProof
			\DisplayProof
			\quad
			\AxiomC{$\sequentPDL{\Gamma}{\AssumsEnv'(\AssumsEnv_1;^{\infty} \AssumsEnv_1)}$}
    		\RightLabel{\IC-R}
			\UnaryInfC{$\sequentPDL{\Gamma}{\AssumsEnv'(\AssumsEnv_1)}$}
			\bottomAlignProof
			\DisplayProof
			\\
			\textrm{\bf Structural rules for Hypersequents:}
			\\
			\AxiomC{$\mathcal{G} \ | \mathcal{H}$}
			\RightLabel{\EE}
			\UnaryInfC{$\ssequentPDL{\mathcal{G}}$}
			\bottomAlignProof
			\DisplayProof
			\quad
			\AxiomC{$\tsequentPDL{}$}
			\RightLabel{\EW}
			\UnaryInfC{$\ssequentPDL{\mathcal{G}}$}
			\bottomAlignProof
			\DisplayProof
			\quad
			\AxiomC{$\ssequentPDL{ \mathcal{G} \ | \ \mathcal{G}  }$}
			\RightLabel{\EC}
			\UnaryInfC{$\ssequentPDL{\mathcal{G}}$}
			\bottomAlignProof
			\DisplayProof
			\\
			%\textrm{\bf Hypersequent specific structural rules:}
			%\\
			\AxiomC{$\sequentPDL{\AssumsEnv_1,\AssumsEnv'_1 }
				{ \AssumsEnv'_3,\AssumsEnv_3}$}
			\AxiomC{$\sequentPDL{\AssumsEnv_2,\AssumsEnv'_2}
				{\AssumsEnv'_4,\AssumsEnv_4}$}
    		\RightLabel{\comM}
			\BinaryInfC{$\csequentPDL{\AssumsEnv_1,\AssumsEnv_2}
				{ \AssumsEnv_3,\AssumsEnv_4}{\AssumsEnv'_1,\AssumsEnv'_2}
				{ \AssumsEnv'_3,\AssumsEnv'_4}$}
			\bottomAlignProof
			\DisplayProof
			%\\
			%\AxiomC{$\sequentPDL{\AssumsEnv_1;^{p}\AssumsEnv'_1 }{\AssumsEnv'_3;^{p}\AssumsEnv_3}$}
			%\AxiomC{$\sequentPDL{\AssumsEnv_2;^{p}\AssumsEnv'_2}{\AssumsEnv'_4;^{p}\AssumsEnv_4}$}
    		%\RightLabel{\comA}
			%\BinaryInfC{$\csequentPDL{\AssumsEnv_1;^{p}\AssumsEnv_2}
			%{ \AssumsEnv_3;^{p}\AssumsEnv_4}{\AssumsEnv'_1;^{p}\AssumsEnv'_2}{ \AssumsEnv'_3;^{p}\AssumsEnv'_4}$}
			%\bottomAlignProof
			%\DisplayProof        
			\\
				\AxiomC{$\sequentPDL{\AssumsEnv_1;^{p}\AssumsEnv'_1 }{\AssumsEnv_3}$}
			\AxiomC{$\sequentPDL{\AssumsEnv_2;^{p}\AssumsEnv'_2}{\AssumsEnv_4}$}
    		\RightLabel{\comA}
			\BinaryInfC{$\csequentPDL{\AssumsEnv_1;^{p}\AssumsEnv_2}
			{ \AssumsEnv_3}{\AssumsEnv'_1;^{p}\AssumsEnv'_2}{ \AssumsEnv_4}$}
			\bottomAlignProof
			\DisplayProof        
			\\
			\textrm{\bf Multiplicatives:}
			\\
			\AxiomC{$\sequentPDL{\AssumsEnv(\emptyset{}_{\times})}{\AssumsEnv'}$}
    		\RightLabel{\oneL}
			\UnaryInfC{$\sequentPDL{\AssumsEnv(1)}{\AssumsEnv'}$}
			\bottomAlignProof
			\DisplayProof
			\quad
			\AxiomC{$\eHyp$}
    		\RightLabel{\oneR}
			\UnaryInfC{$\sequentPDL{\emptyset{}_{\times}}{1}$}
			\bottomAlignProof
			\DisplayProof
			\\
			\AxiomC{$\sequentPDL{\AssumsEnv_1}{\phi}$}
			\AxiomC{$\sequentPDL{\AssumsEnv(\psi)}{\AssumsEnv'_1}$}
    		\RightLabel{\impL}
			\BinaryInfC{$\sequentPDL
			{\AssumsEnv(\AssumsEnv_1 , \phi \ediv \psi )}
			{\AssumsEnv'_1}$}
			\bottomAlignProof
			\DisplayProof
			\quad
			\AxiomC{$\sequentPDL{\AssumsEnv, \phi}{\psi, \AssumsEnv'}$}
    		\RightLabel{\impR}
			\UnaryInfC{$\sequentPDL{\AssumsEnv}{\phi \ediv \psi , \AssumsEnv'}$}
			\bottomAlignProof
			\DisplayProof
			\\
			\AxiomC{$\sequentPDL{\AssumsEnv(\phi, \psi)}{\AssumsEnv'}$}
    		\RightLabel{\monL}
			\UnaryInfC{$\sequentPDL{\AssumsEnv(\phi \conmul \psi)}{ \AssumsEnv'}$}
			\bottomAlignProof
			\DisplayProof
			\quad
			\AxiomC{$\sequentPDL{\AssumsEnv_1}{\phi, \AssumsEnv_3}$}
			\AxiomC{$\sequentPDL{\AssumsEnv_2}{\psi, \AssumsEnv_4}$}
    		\RightLabel{\monR}
			\BinaryInfC{$\sequentPDL{\AssumsEnv_1,\AssumsEnv_2}
			{\phi \conmul \psi , \AssumsEnv_3, \AssumsEnv_4}$}
			\bottomAlignProof
			\DisplayProof
			\\
			\textrm{\bf Additives:}
			\\
			\AxiomC{$\eHyp$}
    		\RightLabel{\topR}
			\UnaryInfC{$\sequentPDL{\emptyset_{+}}{\AssumsEnv(\top)}$}
			\bottomAlignProof
			\DisplayProof
			\quad
			\AxiomC{$\sequentPDL{\AssumsEnv(\emptyset_{+})}{\AssumsEnv'}$}
    		\RightLabel{\topL}
			\UnaryInfC{$\sequentPDL{\AssumsEnv(\top)}{\AssumsEnv'}$}
			\bottomAlignProof
			\DisplayProof
			\quad
			\AxiomC{$\eHyp$}
    		\RightLabel{\botL}
			\UnaryInfC{$\sequentPDL{\AssumsEnv(\bot)}{\AssumsEnv'}$}
			\bottomAlignProof
			\DisplayProof
        	\\
			\AxiomC{$\sequentPDL{\AssumsEnv(\phi;^p \psi)}{\AssumsEnv'}$}
    		\RightLabel{\sandL}
			\UnaryInfC{$\sequentPDL{\AssumsEnv (\phi \phsum{p} \psi)}{\AssumsEnv'}$}
			\bottomAlignProof
			\DisplayProof
			\quad
			\AxiomC{$\sequentPDL{\AssumsEnv_1}{\AssumsEnv(\phi)}$}
			\AxiomC{$\sequentPDL{\AssumsEnv_2}{\AssumsEnv(\psi)}$}
    		\RightLabel{\andR, if $[\Gamma] \leq p$}
			\BinaryInfC{$\sequentPDL{\AssumsEnv_1;^p \AssumsEnv_2}{\AssumsEnv(\phi \phsum{p} \psi) }$}
			\bottomAlignProof
			\DisplayProof
			\\
			\AxiomC{$\sequentPDL{\AssumsEnv(\phi)}{\AssumsEnv_1}$}
			\AxiomC{$\sequentPDL{\AssumsEnv(\psi)}{\AssumsEnv_2}$}
    		\RightLabel{\orL, if $[\Gamma] \leq p$}
			\BinaryInfC{$\sequentPDL{\AssumsEnv(\phi \psum{p} \psi)}{ \AssumsEnv_1;^p \AssumsEnv_2}$}
			\bottomAlignProof
			\DisplayProof
			\\
\AxiomC{$\sequentPDL{\AssumsEnv}{\AssumsEnv'(\phi)}$}
    \RightLabel{\sorRl}
\UnaryInfC{$\sequentPDL{\AssumsEnv}{\AssumsEnv'(\phi \psum{p} \psi)}$}
		\bottomAlignProof
		\DisplayProof
\quad
\AxiomC{$\sequentPDL{\AssumsEnv}{\AssumsEnv'(\psi)}$}
    \RightLabel{\sorRr}
\UnaryInfC{$\sequentPDL{\AssumsEnv}{\AssumsEnv'(\phi \psum{p} \psi)}$}
		\bottomAlignProof
		\DisplayProof
		\\
		\textrm{\bf Quantitative structural rules:}
			\\
			\AxiomC{$\sequentPDL{\AssumsEnv_1 ;^q \AssumsEnv_1 }{\AssumsEnv}$}
			\RightLabel{\pL, $p\leq q$}
			\UnaryInfC{$\sequentPDL{\AssumsEnv_1 ;^p \AssumsEnv_2 }{\AssumsEnv}$}
			\bottomAlignProof
			\DisplayProof
			\quad
			\AxiomC{$\sequentPDL{\AssumsEnv }{\AssumsEnv_1 ;^q \AssumsEnv_2}$}
			\RightLabel{\pR, $p\leq q$}
			\UnaryInfC{$\sequentPDL{\AssumsEnv }{\AssumsEnv_1 ;^p \AssumsEnv_2}$}
			\bottomAlignProof
			\DisplayProof
			\end{empheq}
	\end{spreadlines}}
	\vspace*{-1em}
	
	\caption{\emph{\footnotesize{Propositional sequent calculus for \OL{} 
	}}}
	\label{fig:seq-rules}
	\vspace*{-1.5em}
\end{figure}

\subsection{Semantics and Soundness}
\label{section:soundness}
We build up to the notion of validy of hypersequents, respect to which we will prove soundness of \OL{}. We start with \OL{}'s semantics and defining equation, following the substructural logic tradition \citep{galatos2007residuated},

\begin{definition}[Formula Validity]
    The formula $\phi$ is \emph{valid} iff $1 \leq \m{\phi}$, where $\m{\phi}$ is defined inductively as follows:
    \begin{equation}
\label{semantics}
    \begin{split}
    &\m{\unit} := 1 \quad \m{\top} := \infty \quad \m{\bot} := 0\\
    &\m{\ldual{\phi}} := \edual{\m{\phi_{1}}}\\
    &\m{\phi_{1} \ediv \phi_{2}} := \m{\phi_{1}} \ediv \m{\phi_{2}}\\
    &\m{\phi_{1} \conmul \phi_{2}} := \m{\phi_{1}} \conmul \m{\phi_{2}}\\
    &\m{\phi_{1} \dismul \phi_{2}} := \m{\phi_{1}} \dismul \m{\phi_{2}}\\
    &\m{\phi_{1} \psum{p} \phi_{2}} := \m{\phi_{1}} \psum{p} \m{\phi_{2}}\\
    &\m{\phi_{1} \phsum{p} \phi_{2}} := \m{\phi_{1}} \phsum{p} \m{\phi_{2}}\\
    \end{split}
\end{equation}
\end{definition}

And introduce some notable valid formulae,

\begin{lemma}[Some Valid Formulas]
\label[Lemma]{SomeValidFormulae}
    The following are valid formulae in \OL{}:
    \begin{enumerate}
        \item \textbf{Additive Prelinearity.} $(\phi_1 \ediv \phi_2 ) \psum{p}  (\phi_2 \ediv \phi_1)$,
        \item \textbf{Multiplicative Prelinearity.} $(\phi_1 \ediv \phi_2 ) \conmul{}  (\phi_2 \ediv \phi_1)$,
    \end{enumerate}
\end{lemma}

Next we define validity of sequents,

\begin{definition}[Sequent Validity]
\label[definition]{sequentValidity}
 The sequent $\sequentPDL{\AssumsEnv}{ \AssumsEnv'}$ is \emph{valid} iff 
$1 \leq \m{\AssumsEnv}_a \ediv{} \m{\AssumsEnv'}_c$, where  $\m{\AssumsEnv}_a $ and $\m{\AssumsEnv'}_c$ are defined inductively as follows:

\begin{enumerate}
\item For antecedents:
\begin{equation}
    \begin{split}
    \m{\emptyset{}_{\times}}_a = 1 &\quad \m{\emptyset{}_{+}}_a = \infty\\
     \textrm{If \ } \Gamma \equiv \Gamma_1, \Gamma_2 & \textrm{\ then \ } 
     \m{\AssumsEnv}_a = \m{\Gamma_1}_a \conmul \m{\Gamma_2}_a \\
          \textrm{If \ } \Gamma \equiv \Gamma_1;^p \Gamma_2 & 
          \textrm{\ then \ } \m{\AssumsEnv}_a = \m{\Gamma_1}_a \phsum{p} \m{\Gamma_2}_a \\
    \end{split}
\end{equation}
\item For consequent:
%\knote{TO-DO below: replace $\Delta$ with $\AssumsEnv$}
\begin{equation}
    \begin{split}
    \m{\emptyset{}_{\times}}_c = 1 &\quad \m{\emptyset{}_{+}}_c = 0\\
     \textrm{If \ } \AssumsEnv \equiv \AssumsEnv_1, \AssumsEnv_2 & 
     \textrm{\ then \ }  \m{\AssumsEnv}_c =  \m{\AssumsEnv_1}_c 
     \dismul \m{\AssumsEnv_2}_c  \\
     \textrm{If \ } \AssumsEnv \equiv \AssumsEnv_1;^p \AssumsEnv_2 & \textrm{\ then \ }  
     \m{\AssumsEnv}_c =  \m{\AssumsEnv_1}_c \psum{p} \m{\AssumsEnv_2}_c \\
    \end{split}
\end{equation}
\end{enumerate}
%\knote{The above definition is given parametrically on $p$, But the below theorem suggests how it can be chosen in practice, per proof}
The base case for both $\m{\AssumsEnv}_a$ and $\m{\AssumsEnv}_c$ is given by $\AssumsEnv$ being formulae, with interpretation as defined in \cref{semantics}.

\end{definition}

Note $\phi_1 \ediv{} \phi_2$ is valid iff $\sequentPDL{\phi_1}{ \phi_2}$. 

\begin{comment}
\begin{lemma}[Monotony of Bunch Semantics]
\label{Monotony of Bunch Semantics}
\,
\begin{enumerate}
    \item If $\m{\AssumsEnv_1}_a \leq \m{\AssumsEnv_2}_a$, then 
    $\m{\AssumsEnv(\AssumsEnv_1)}_a \leq \m{\AssumsEnv(\AssumsEnv_2)}_a$.
    
    \item If $\m{\AssumsEnv_1}_c \leq \m{\AssumsEnv_2}_c$, then 
    $\m{\AssumsEnv(\AssumsEnv_1)}_c \leq \m{\AssumsEnv(\AssumsEnv_2)}_c$.
    
    \item If $\m{\AssumsEnv_1}_a \leq \m{\AssumsEnv_2}_c$, then 
    $\m{\AssumsEnv(\AssumsEnv_1)}_a \leq \m{\AssumsEnv(\AssumsEnv_2)}_c$.
\end{enumerate}
\end{lemma}

\begin{corollary}[Monotonicity of PDL Validity]
    If $\sequentPDL{\AssumsEnv'_1}{\AssumsEnv'_2}$ is valid, then 
    $\sequentPDL{\AssumsEnv(\AssumsEnv'_1)}{\AssumsEnv(\AssumsEnv'_2)}$ is 
    valid.
\end{corollary}
\end{comment}

\begin{lemma}[Some Properties of Bunches]
\label[lemma]{SomePropertiesOfBunches}
\,
    \begin{enumerate}
       % \item \textbf{Reflexivity.} If $\AssumsEnv_1 \not \equiv \emptyset_{+}$ then 
    %${\sequentPDL{\AssumsEnv_1}{\AssumsEnv_1}}$ is valid.
        \item \textbf{Monotonicity.}
    \begin{enumerate}
    \item If $\m{\AssumsEnv_1}_a \leq \m{\AssumsEnv_2}_a$, then 
    $\m{\AssumsEnv(\AssumsEnv_1)}_a \leq \m{\AssumsEnv(\AssumsEnv_2)}_a$.
    
    \item If $\m{\AssumsEnv_1}_c \leq \m{\AssumsEnv_2}_c$, then 
    $\m{\AssumsEnv(\AssumsEnv_1)}_c \leq \m{\AssumsEnv(\AssumsEnv_2)}_c$.
    
    %\item If $\sequentPDL{\AssumsEnv_1}{\AssumsEnv_2}$ is valid, then $\sequentPDL{\AssumsEnv(\AssumsEnv_1)}{\AssumsEnv(\AssumsEnv_2)}$ is valid.
\end{enumerate}
    \item \textbf{Lax-linearity.}
    \begin{enumerate}
     %\item If $1 \leq \m{\AssumsEnv_1}_a$, then $\m{\AssumsEnv(\AssumsEnv_1,\AssumsEnv_2 )}_a \leq \m{\AssumsEnv_1, \AssumsEnv(\AssumsEnv_2)}_a$.
    
    %\item If $1 \leq \m{\AssumsEnv_1}_c$, then $\m{\AssumsEnv(\AssumsEnv_1,\AssumsEnv_2 )}_c \leq \m{\AssumsEnv_1, \AssumsEnv(\AssumsEnv_2)}_c$.
    \item $\m{\AssumsEnv(\AssumsEnv_1;^{p}\AssumsEnv_2 )}_a \geq \m{\AssumsEnv(\AssumsEnv_1);^{p} \AssumsEnv(\AssumsEnv_2)}_a$
    \item $\m{\AssumsEnv(\AssumsEnv_1;^{p}\AssumsEnv_2 )}_c \leq \m{\AssumsEnv(\AssumsEnv_1);^{p} \AssumsEnv(\AssumsEnv_2)}_c$
    %\item ${\sequentPDL{\AssumsEnv(\AssumsEnv_1;^{p}\AssumsEnv_2)} {\AssumsEnv(\AssumsEnv_1);^{p}\AssumsEnv(\AssumsEnv_2)}}$ is valid.
    %\item If $\AssumsEnv_1 \not \equiv \emptyset_{+}$ and $\AssumsEnv_2 \not \equiv \emptyset_{+}$ then ${\sequentPDL{\AssumsEnv(\AssumsEnv_1,\AssumsEnv_2)} {\AssumsEnv_1,\AssumsEnv(\AssumsEnv_2)}}$ is valid.
    %\item  ${\sequentPDL{\AssumsEnv(\AssumsEnv_1;^{p}\AssumsEnv_2)}
    %{\AssumsEnv(\AssumsEnv_1);^{p}\AssumsEnv(\AssumsEnv_2)}}$ is valid.
    \end{enumerate}
    \end{enumerate}
\end{lemma}

%\jnote{Should I include proofs?}


\begin{comment}
\begin{lemma}[Monotoncity of Bunches]
    \label[lemma]{MonotonicityOfBunches}
    If $\sequentPDL{\AssumsEnv'_1}{\AssumsEnv'_2}$ is valid, then 
    $\sequentPDL{\AssumsEnv(\AssumsEnv'_1)}{\AssumsEnv(\AssumsEnv'_2)}$ is 
    valid.
\end{lemma}
\begin{proposition}[Bunch Semantics are Lax Linear]
\label{Bunch Semantics are Lax Linear}
\,
\begin{lemma}
    \item If $1 \leq \m{\AssumsEnv_1}_a$, then 
    $\m{\AssumsEnv(\AssumsEnv_1,\AssumsEnv_2 )}_a \leq 
    \m{\AssumsEnv_1, \AssumsEnv(\AssumsEnv_2)}_a$.
    
    \item If $1 \leq \m{\AssumsEnv_1}_c$, then 
    $\m{\AssumsEnv(\AssumsEnv_1,\AssumsEnv_2 )}_c \leq 
    \m{\AssumsEnv_1, \AssumsEnv(\AssumsEnv_2)}_c$.
    
    \item If $\m{\AssumsEnv_1}_a \leq \m{\AssumsEnv_1}_c$ and 
    $\m{\AssumsEnv_2}_a \leq \m{\AssumsEnv_2}_c$, then 
    $\m{\AssumsEnv(\AssumsEnv_1,\AssumsEnv_2 )}_a \leq 
    \m{\AssumsEnv_1, \AssumsEnv(\AssumsEnv_2)}_c$.
\end{lemma}
\end{proposition}
\begin{lemma}[Lax Linearity of Bunches]
\label[lemma]{BunchesAreLaxLinear}
    If $\AssumsEnv_1 \not \equiv \emptyset_{+}$ 
    and $\AssumsEnv_2 \not \equiv \emptyset_{+}$ then 
    ${\sequentPDL{\AssumsEnv(\AssumsEnv_1,\AssumsEnv_2)}
    {\AssumsEnv_1,\AssumsEnv(\AssumsEnv_2)}}$ is valid.
\end{lemma}
\end{comment}



Lastly, we define validity of hypersequents, 

\begin{definition}[Hypersequent Validity]
    We say a hypersequent $\Hyp$ is \emph{valid} iff $1 \leq \m{\Hyp{}}$, where $\m{\Hyp{}}$ is defined inductively as follows: if $\Hyp \equiv \Hyp{}_1 | \Hyp{}_2$ then $\m{\Hyp} = \m{\Hyp{}_1} \psum{\infty} \m{\Hyp{}_2}$, and the base case is given by $\Hyp$ being a formulae.
\end{definition}

Therefore, ($|$) can be seen an additive conjunction. We can now state soundness,

\begin{theorem}[Soundness of \OL{}]
    If a hypersequent is provable in \OL{}, then it is valid.
\end{theorem}

\begin{proof}
    The proof proceeds by induction on the length of the proof and case analysis on rule shape.
    
    \textbf{Equivalence of Bunches.}
    We must first prove that if $\Gamma_1 \Leftrightarrow \Gamma_2$ then $\m{\Gamma_1}_a = \m{\Gamma_2}_a$ and $\m{\Gamma_1}_c = \m{\Gamma_2}_c$. This follows directly from  \cref{IsLattice}. $\LJAxiom{}$ follows by induction over the shape of $\Gamma$. 

    \textbf{Initial hypersequents.} It is immediate that
    $\m{\eHyp{}} = \m{\sequentPDL{\emptyset{\times}}{\emptyset{\times}}} = 1 \ediv{} 1 = 1$.
     
    \textbf{Basic structural rules.} $\Equi$ and $\Ass$ are trivial. $\IW$ follows from conjunctive and disjunctive monotonicity (\cref{SomePropertiesOfP}) and monotonicity of bunches (\cref{SomePropertiesOfBunches}). $\IC$ follows from additive collapse (\cref{AdditiveCollapse}) and monotonicity of bunches.

    \textbf{Structural rules for Hypersequents.} $\EE$, $\EW$ and $\EC$ follow from properties of the $\max$ function. We prove $\comM$ by contradition. Let us then have the following hypothesis: $a \conmul{} A \leq c \dismul{} C$, $b \conmul{} B \leq d \dismul{} D$, $c \conmul{} d < a \conmul{} b$, and $C \conmul{} D < A \conmul{} B$. Therefore, 
    $a \conmul{} A \conmul{} b \conmul{} B \leq c \dismul{} C \dismul{} d \dismul{} D$ and $a \conmul{} A \conmul{} b \conmul{} B > c \dismul{} C \dismul{} d \dismul{} D$,
    which is a contradiction. For $\comA{}$, we have that $1 \leq (a\phsum{p}b \ediv{} e)$ and $1 \leq (c \phsum{p}d \ediv{} f)$. Hence,
    $$1 \leq ((a\phsum{p}b) \ediv{} e) \phsum{\infty} ((c \phsum{p}d) \ediv{} f)$$
    $$1 \leq ((\edual{a} \dismul{} e) \psum{p} (\edual{b} \dismul{} e)) \phsum{\infty} ((\edual{c} \dismul{} f) \psum{p} (\edual{d} \dismul{} f))$$
    Notice that by sub-distributivity,
    $$
    (\edual{b}\dismul{}e) \phsum{\infty} (\edual{c}\dismul{}f) \leq (\edual{b}\dismul{}f) \psum{\infty} (\edual{c}\dismul{}e)
    $$
    Then by monotonicity and semi-additivity,
    \begin{equation}
        \begin{split}
            & ((\edual{a} \dismul{} e) \psum{p} (\edual{b} \dismul{} e)) \phsum{\infty} ((\edual{c} \dismul{} f) \psum{p} (\edual{d} \dismul{} f)) \leq \\
            & ((\edual{a} \dismul{} e) \psum{p} (\edual{b} \dismul{} e)) \psum{\infty} ((\edual{c} \dismul{} f) \psum{p} (\edual{d} \dismul{} f))
        \end{split}
    \end{equation}
We conclude,
$$
1 \leq ((a\phsum{p}c) \ediv{} e) \psum{\infty} ((b \phsum{p}d) \ediv{} f)
$$
    \textbf{Multiplicatives.} $\oneR$, $\oneR$, $\impR$, $\monL$ and $\monR$ are immediate. For $\impL$ we have the hypothesis $a \leq b$ and $f(c) \leq d$, where $f = \m{\Gamma(-)}$. Then, by monotonicity of bunches (\cref{SomePropertiesOfBunches})
    \begin{equation}
        \begin{split}
            & b \ediv{} a \leq 1\\
            & (b \ediv{} a) \conmul{} c \leq c\\
            & f(b \ediv{} (a \conmul{} c)) \leq f(c)\\
            & f(a \conmul{} (b \ediv{} c)) \leq d\\
        \end{split}
    \end{equation}
    
    \textbf{Additives.} $\topL$, and $\sandL$ are immediate. $\topR$ and $\botL$ follow from induction over the shape of $\Gamma$. $\andR$ and $\orL$ follow from semi-additivity (\cref{SomePropertiesOfP}) and lax-linearity of bunches (\cref{SomePropertiesOfBunches}). $\sorRl$ and $\sorRr$ follow from semi-additivity (\cref{SomePropertiesOfP}) and bunches being monotonic increasing (\cref{SomePropertiesOfBunches}).

    \textbf{Quantitative structural rules.} Lastly, $\pL$ and $\pR$ follow from p-monotonicity (\cref{SomePropertiesOfP}). 

\end{proof}

%\subsection{First-Order Extension Progress}

%\subsection{Formalization Progress}



%% ----------------------------------------------------------------
%% NextSteps.tex
%% ---------------------------------------------------------------- 
\section{Next Steps} \label{section:Next Steps}

\TODO


%%% ----------------------------------------------------------------
%% Conclusions.tex
%% ---------------------------------------------------------------- 
\section{Conclusions} \label{section: Conclusions}

\TODO


%\begin{lstlisting}[caption=Listing of what an example listing would be like]
%This is a test listing

%The test listing has serveral lines
%to show how the listings
%will be displayed
%\end{lstlisting}
%\appendix
%\include{AppendixA}
%\backmatter
%\chapter{Glossary [if relevant]}
\bibliographystyle{plainnat}
\bibliography{UOS}
%\chapter{Bibliography}
%To use bibliography as well as the references section use the \texttt{multibbl} package.
%\chapter{Index [if relevant]}

\begin{comment}
\section{Appendix}

\begin{figure}[H]
\label{Gantt}
\centering
  \includegraphics[width=1.15\columnwidth]{Figures/Gantt.pdf}
  \caption{Gantt of our plan until the second progression review.}
\end{figure}
\end{comment}

\end{document}
%% ----------------------------------------------------------------
