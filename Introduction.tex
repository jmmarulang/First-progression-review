%% ----------------------------------------------------------------
%% Introduction.tex
%% ---------------------------------------------------------------- 
\chapter{Introduction} \label{Chapter:Introduction}

%\jnote{Not sure if I should frame the text around symbolic computing vs subsymbolic computing (intersymbolic computing, Symbolic-numeric computating) or around symbolic vs subsymbolic AI (intersymbolic AI, neuro-symbolic AI). Will focus on AI for now}

%\jnote{Inspired on \cite{Platzer_2024}. The terms \SuAI{}, \SiAI{} and \InAI are taken from there.}

Broadly speaking, there exists  two very distinct approaches to \emph{\AILong{}}  (\emph{\AI{}}): One rooted in reasoning and another in learning \mcita{}.  In \citetalias{Platzer_2024}, \citeauthor{Platzer_2024} proposes the terms \emph{\SiAI{}} and \emph{\SuAI{}}, respectively. Here we define them a bit more generally. \SiAI{} (also referred to as good old fashioned AI \mcita{}, classical AI \mcita{},  or logic-based AI \mcita{} ) relates to algebraic computing, and emphasizes finding analytical solutions by manipulating logical expressions. This approach, by principle, prioritizes interpretability and preserving meaning. Current relevant examples include SMT/SAT solvers \mcita{}, theorem provers \mcita{},  as well as many \emph{programming languages} (PL) \mcita{}.   \SiAI{} has had a broad impact on planning \mcita{}, gameplay \mcita{} and software/hardware verification \mcita{}, as well as on the field of mathematics \mcita{}.  On the other hand, \SuAI{} (also referred to as pattern engines \mcita{})
%\jnote{I like the term "universal approximators", based on the universal approximation theorem that states that a neural network can approximate any continuous function to any desired degree of accuracy. I feel it better portrais what subsymbolic AI is, and would like to add it to the list of alternative names. But as far as I know its not been introduced before.}
relates to numerical computing,  and focuses on approximate solutions by applying statistical and optimization methods. They are often data driven, and do not require an explicit algorithm to operate (beyond the indirect computations performed to approximate the result). The more relevant examples of \SuAI{} are \emph{Machine Learning} (ML) \mcita{} and \emph{Reinforcement Learning} (RL) \mcita{}, but older methods such as Kalman Filters \mcita{} could arguably also fall under this category.  \SuAI{} has recently grown in quality and proliferated to many applications such as image/language processing \mcita{}, and simulation \mcita{}. 

In order to leverage their respective strengths, there is a growing interest in studying and developing methods that merge Symbolic and Subsymbolic AI \mcita{}. This broad category, nicked \emph{\InAI{}} by \citeauthor{Platzer_2024} (also referred to as neuro-symbolic AI \mcita{}, or hybrid intelligent systems \mcita{}), can range from applying logical principles in the architecture of \emph{ \NN{}s} (NN)  \mcita{}, to using large language models to generate better heuristics for theorem provers \mcita{}. Even more, \InAI{} has found particular success in cyber-physical systems, where it often plays the role of a controller \citep{Platzer_2024}. 

Despite this, many \InAI{} systems often lack formal rigour, which prevents us from taking full advantage of its components. From the \SiAI{} perspective,  many symbolic principles are often haphazardly applied \mcita{}. Even more, many \InAI{} systems lack logical, algebraic or categorical representations, with proper syntax and semantics (for examples, see \mcita{}). Therefore, the interpretability and theoretical guarantees that symbolic systems benefit from ends up being obfuscated. This is also exacerbated when dealing with \NN{}, known to be black boxes. On the other hand, the statistical and probabilistic guarantees that are given for certain \SuAI{}  systems (such as robustness \mcita{} and generalization boundaries \mcita{} ), as well as the effects that \SiAI{} may have on performance remain understudied for \InAI{}.


\yada


\TODO
