%% ----------------------------------------------------------------
%% Introduction.tex
%% ---------------------------------------------------------------- 
\section{Problem Statement} \label{section:Problem Statement}

\textbf{The two flavours of Artificial Intelligence.}
Broadly speaking, there exists  two very distinct approaches to \emph{\AILong{}}  (\emph{\AI{}}): One rooted in reasoning and another in learning \citep{platzer2024intersymbolic}.  \SiAI{} (also referred to as good old fashioned AI \citep{haugeland1989artificial}, classical AI \citep{garnelo2019reconciling},  or logic-based AI \citep{thomason2003logic}) relates to algebraic computing, and emphasizes finding analytical solutions by manipulating logical expressions. This approach, by principle, prioritizes interpretability and preserving meaning. Current relevant examples include \emph{SMT/SAT solvers} \citep{barrett2018satisfiability, alyahya2022structure}, \emph{theorem provers} \citep{bartek2025vampire, barras1999coq},  as well as many \emph{programming languages} (PL) \citep{korner2022fifty, perkel2019julia, klabnik2023rust}.
%\jnote{Not sure about what to reference for programming languages} 
\SiAI{} has had a broad impact on planning \citep{geffner2013concise}, gameplay \citep{newell1958chess} and system verification \citep{leroy2016compcert, tihanyi2025new}, as well as on the field of mathematics \cite{Blokpoel2024}.  On the other hand, \SuAI{} (also referred to as pattern engines \citep{julia2020there})
%\jnote{I like the term "universal approximators", based on the universal approximation theorem that states that a neural network can approximate any continuous function to any desired degree of accuracy. I feel it better portrais what subsymbolic AI is, and would like to add it to the list of alternative names. But as far as I know its not been introduced before.}
relates to numerical computing,  and focuses on approximating solutions by applying statistical and optimization methods. They are often data driven, and do not require an explicit algorithm to operate (beyond the indirect computations performed to approximate the result). The more relevant examples of \SuAI{} are \emph{Deep Learning} \citep{norvig2002modern} and \emph{Reinforcement Learning} \citep{sutton1998reinforcement}, but older methods such as \emph{Kalman Filters} \citep{simon2001kalman} or \emph{Monte Carlo Simulation} \citep{martin2024computing} could arguably also fall under this category.  \SuAI{} has recently grown in quality and proliferated to many applications such as image/language processing \citep{thapa2024application, vaswani2017attention}, and simulation \citep{jumper2021highly}. 

\textbf{\InAI{}.} In order to leverage their respective strengths, there is a growing interest in studying and developing methods that merge Symbolic and Subsymbolic AI. This broad category, nicked \emph{\InAI{}}  \citep{platzer2024intersymbolic} (also referred to as neuro-symbolic AI \citep{d2009neural}, or hybrid intelligent systems \citep{medsker2012hybrid}), can range from applying logical principles on the architecture of \emph{ \NN{}s} (NN)  \citep{badreddine2022logic, petersen2022deep}, to using \SuAI{} to generate better heuristics for theorem provers \citep{laurent2022learning}. On top of this, \InAI{} has found particular success in cyber-physical systems, where it often plays the role of a controller \citep{platzer2024intersymbolic}.

\textbf{\DL{}.} One of the main challenges of \InAI{} is the grounding of symbolic knowledge into numerical representations. A promising approach is the study of \emph{\QL{}s} (QL), i.e.~logics that model the real numbers. They have been studied for decades, and date back to the ideas of Kleene, G\"{o}del, and Łukasiewicz at the start of the 20th century \citep{cintula2011handbook, prooffuzzy}. Some relevant QLs include \emph{Fuzzy Logics} \citep{cintula2011handbook} and the \emph{Logics of The Lawvere Quantile} \citep{bacci2024polynomial, bacci2023propositional, capucci2024quantifiers, bacci2025induction}. To illustrate QLs, let us have a toy syntax with atomic propositions and conjunction, such as
\begin{equation}
\begin{split}
    \Phi \ni \phi &:= A \,|\, \phi \land \phi
\end{split}
\end{equation}
where $\phi$ is interpreted through a mapping $\tempty{\cdot} : \Phi \rightarrow D$ such that $ \tempty{\phi} \in D \subseteq \Ereal$. $D$ varies among logics and restricts the interpretation of connectives. For example, the
G\"{o}del logic \citep{BAAZ200723} has a standard semantics over $[0, 1]$ where the conjunction is interpreted as the minimum function. \emph{\DL{}s} (DL) form a family of methods that apply key insights from QLs to \InAI{}. DLs have been applied on property-driven training \citep{FLINKOW2025103280}, safe-by-construction systems \citep{badreddine2022logic}, and SAT solving \citep{kyrillidis2021continuous, gaglione2022maxsat}. 

\textbf{What makes a good DL.} To function as a bridge between symbolic and subsymbolic AI, a DL would benefit from certain properties. From the \SiAI{} perspective, a DL should be expressive enough to encode properties of interest (e.g. see \citep{vehicle}). Even more, to certify its implementation, a DL should possess a deductive system, as well as some form of soundness and completeness proofs with respect to a semantics \citep{floyd1993assigning, goguen1977initial}. From the \SuAI{} perspective, differentiability of its interpretation is an obvious candidate; continuity or convexity are also widely considered desirable; \citeauthor{varnai2020robustness} suggest characterizing DLs in terms of their \textit{geometric properties} \citep{varnai2020robustness}. Out of these, the most notable is \emph{Shadow-lifting}:

\begin{definition}[Shadow-lifting]
    \label[Definition]{Shadow-lifting}
    A logical operator $A : \Phi ^ n \rightarrow \Phi$ satisfies shadow-lifting if, for any $\tempty{\phi} \neq 0$ and $i \in [1,n]$,
    \begin{equation*}
	\left. \dfrac{\partial \tempty{A(\phi_1, ..., \phi_n)}}{\partial \tempty{\phi_i}}\right\rvert_{\phi_j = \phi \text{ where } i \neq j} >0
	\end{equation*}
	where $ \partial $ denotes partial differentiation.
\end{definition}

%\jnote{Is this equivalent to saying that the gradient of $A$ is always positive component-wise? It is NOT but its strangely similar. The only difference is that in shadow-lifting, the partial derivative must only be positive if every other component has been replaced by the SAME constant. Which seems an oddly specific condition to me. I feel this definition could be generalized.}

However, it has been shown that a an operator cannot meet shadow-lifting while being associative and idempotent \citep{varnai2020robustness}. \citeauthor{van2022analyzing} also mention three gradient problems to be avoided that commonly arise from logic operators: \emph{single-passing}, \emph{vanishing}, and \emph{exploding gradients} \citep{van2022analyzing}. Moreover, it is not well understood the effects that DLs have on performance \citep{flinkow2025generalised}, as well as on the statistical and probabilistic guarantees of certain systems, such as generalization error boundaries \citep{jakubovitz2019generalization}.

%\jnote{Mention shadow lifting doesnt allow assoc or idemp}

\textbf{The quantifier problem.} Nevertheless, there is one fundamental problem that DLs face: Many properties of interest for machine learning involve quantifiers, yet the majority of QLs are propositional \citep{bacci2024polynomial, bacci2023propositional, bacci2025induction}. A canonical specification of this kind is  \textit{robustness} \citep{casadio2022neural},  i.e. small perturbations to the inputs of a neural network should result in small changes to its output, formally:
\begin{definition}[$\epsilon$-$\delta$-Robustness] % no space here
\label{Robustness}%
    Let $\epsilon, \delta \in \real^+$, $||\cdot||$ be a norm, and $f : \real^n \rightarrow \real^m$ be a measurable function.
    One says \textit{$f$ is $\epsilon$-$\delta$-robust} around $\bar x \in \real ^ n$ if 
    \begin{equation}
    \label{eq:robustness}
        \forall x\in \real^n , ||x - \bar x|| \leq \epsilon \Rightarrow || 
			f(x) - f(\bar x)|| \leq \delta  
    \end{equation}
\end{definition}

Expanding some sound and complete propositional QLs to first-order logic often comes at the expense of either completeness or continuity \citep{cintula2011handbook, slusarz2023logic}.  
%For example, the first-order extension of Gödel logic is the only one, among the most prominent fuzzy logics \mcita{}, that is sound and complete w.r.t. models with values in $[0,1]$ and with universal and existential quantifiers interpreted as infima and suprema \mcita{}.However, connectives of this logic are not continuous and therefore not suitable for gradient-descent algorithms.

\textbf{A novel approach.} Given the provided information, can we develop a first-order DL while maintaining desirable properties?   Recently, a promising approach for first-order QLs was proposed by \citeauthor{capucci2024quantifiers}: interpreting quantifiers as \textit{generalized means}, while introducing a "softness" modality that balances shadow-lifting and idempotence \citep{capucci2024quantifiers}. We build on these ideas to develop a novel well-behaved DL, that we nick \emph{\OurLogic} (\OL). Unlike \citeauthor{capucci2024quantifiers}, who uses a deep-inference inspired framework \citep{guglielmi2007system, guglielmi2015deep}, our approach follows the subtructural logic tradition \citep{galatos2007residuated}, taking elements from both the logic of Bunched Implications (BI) \citep{o1999logic} (an extension of linear logic \citep{Wadler1993, agliano2025algebraic}) and fuzzy logics \citep{cintula2011handbook, prooffuzzy}. We leverage this to give algebraic semantics for our DL, while studying its relation to other substructural logics and \citeauthor{capucci2024quantifiers}'s logic. 
To provide assurance of the correctness of our results, we aim to mechanise our proofs in Rocq, making use of its Mathematical Components library (\mathcomp{}) \cite{mathcomp}. We seek this formalization to be a stepping stone for the development of programming language support for verification of \InAI{} \citep{vehicle}. 

\subsection{Research Requirements}
\label{ResearchRequirements}
The preceding introduction highlights two groups of properties desirable for DLs: \emph{Symbolic} and \emph{Subsymbolic} properties. 

\textbf{Symbolic properties.} For our logic to be expressive, it should possess negation, conjunction and disjunction operators. Negation should be \emph{involutive}; conjunction and disjunction should be \emph{commutative}, \emph{associative} and \emph{idempotent}; and all operators should \emph{distribute} in the classical manner \citep{galatos2007residuated}. If a logic meets all previous properties we say it is \emph{compositional}. Similarly, we would like our universal quantifiers to be \emph{aggregation operators} \citep{LIU19981}, and existencials to be their duals \citep{LIU19981}. If a logic meets these properties, we say it is \emph{aggregative}.  We also aim for our DL to be \emph{sound} and \emph{complete}. Intuitively, a logic is sound if any sentence that is provable in its deductive system is also true on its semantics \citep{galatos2007residuated}. Conversely, a logic is complete if any sentence that is true on its semantics is also provable in its deductive system \citep{galatos2007residuated}. While soundness provides theoretical guarantees of correctness, completeness is needed to guarantee that the syntax and the semantics of the logic match.

\textbf{Subsymbolic properties.} The interpretation of our logic should be \emph{differentiable}, 
%\jnote{Should we focus on almost everywhere differentiable functions instead? It is more general and the minimum and maximum fall under this category (infimum does not). Its what they use to analyse DL2 in \cite{fischer2019dl2}.}
and meet the following geometric properties proposed by \citeauthor{varnai2020robustness}: \emph{scale-invariance}, \emph{weak smoothness} and \emph{shadow-lifting} \citep{varnai2020robustness}. \citeauthor{van2022analyzing} mention three gradient problems to be avoided that commonly arise from logic operators: \emph{single-passing}, \emph{vanishing gradients}, and \emph{exploding gradients} \citep{van2022analyzing}. All previous properties are useful for optimization.  

\subsection{Research Questions}

Our high-level goal is to develop and study a DL that meets the symbolic and subsymbolic properties. To this aim, we divide our research into six research questions:

\begin{enumerate}
    \item \textbf{Sound propositional DL.} Can we develop a sound, compositional, and propositional DL, while maintaining subsymbolic properties? 
    
    Our current version of \OL{} could be summarized as a "soft" bunched logic \citep{o1999logic} with hypersequents from fuzzy logic \citep{prooffuzzy}. Like \citeauthor{capucci2024quantifiers}, our DL makes use of a modality to balance shadow-lifting and compositionality. This modality is also applied to bunches. Our DL has been proven sound with respect to its numerical interpretation. We are currently researching how to develop algebraic semantics.

    %\jnote{Include formalization intent somewhere in here? How to state it as a research question?}

    \item \textbf{Relation to other logics.} How does \OL{} relate to other logics? 
    
    Being an extension of linear logic \citep{Wadler1993, agliano2025algebraic}, \OL{} (or at least a subset of it) belongs to the family of substructural logics \citep{galatos2007residuated}, with its monoidal operator resembling that of product logic \citep{cintula2011handbook, prooffuzzy}. It remains unclear where in the substructural family it would be positioned. As for \citeauthor{capucci2024quantifiers}'s logic, they very much resemble each other, and possess the same numerical interpretation. However, each logic possesses a different definition of validity, and therefore of soundness and completeness \citep{galatos2007residuated, capucci2024quantifiers}. We expect this to affect the algebraic semantics as well, beyond just changing its defining equation \citep{galatos2007residuated, agliano2025algebraic}. 
    
    \item \textbf{Complete propositional DL.} Can we prove propositional \OL{} complete?
    
    A first approximation to completeness is to prove that a collection of axioms that are true in the semantics, are provable (e.g. prelinearity \citep{prooffuzzy}). While this does not guarantee completeness, it is an informative necessary condition. Then, we hope to leverage \OL{}'s relation to substructural logics to prove completeness by applying known techniques from fuzzy logic \citep{cintula2011handbook, galatos2007residuated}. It remains unclear if this is possible. 
    
    \item \textbf{Sound predicate DL.} Can \OL{} be extended into a sound aggregative logic, while maintaining subsymbolic properties?

    Extending into first-order requires modifying our numerical interpretation, and therefore our semantics. For our current approach to maintain soundness, additional conditions must be imposed on substitution. On the other hand, it remains unclear if subsymbolic properties are maintained. 
    
    \item \textbf{Complete predicate DL.} Can we prove our predicate \OL{} complete?
    
    Similar approach to the propositional case. 
    
    \item \textbf{Practicality.} Does propositional/predicate \OL{} offer some insight on property-driven training?

    We would like to study the effects \OL{} on the performance and generalisability \citep{jakubovitz2019generalization} of its implementations. This requires both theoretical and experimental evidence. 
    
\end{enumerate}
