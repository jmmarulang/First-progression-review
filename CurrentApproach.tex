%% ----------------------------------------------------------------
%% CurrentApproach.tex
%% ---------------------------------------------------------------- 
\section{Background and Preliminary Results} \label{section:CurrentApproach}

In this section we introduce some relevant background theory and summarize some preliminary results. We introduce a propositional version of \OL{} by its syntax, semantics, and sequent calculus. We then mention our progress in formalization.

\subsection{Preliminaries and Notation}
\label{Preliminaries}
%\subsubsection{Logic of the Reals}
%\subsubsection{\citeauthor{capucci2024quantifiers}'s logic}
We introduce preliminaries from the extended arithmetic of the reals. They are a modified version of \cite{capucci2024quantifiers}. We diverge from \emph{ibid.} in notation. Our base setting are the positive extended reals $[0,\infty]$. %, considered as sup-lattice with the usual order $\leq$.
%The topology on $\real^+$ is extended to $[0,\infty]$ by adding to the opens all the intervals $(a, \infty]$.
%As a measure space, $[0,\infty]$ is considered equipped with completion of its Borel $\sigma$-field (i.e. the Lebesgue $\sigma$-field); and then further equipped with the obvious extension of the Lebesgue measure given by setting $\lambda((a,\infty]) = \infty$ for $a < \infty$ and $\lambda(\{\infty\})=0$.

\begin{definition}[$p$-Sum]
\label{$p$-Sum}
    %On $[0,\infty]$, 
    \emph{$p$-sum} and \emph{harmonic $p$-sum} are, respectively, the following operations:
    \begin{equation*}
		\begin{tabular}{c|ccc}
			$a \psum{p} b$ & $0$ & $a \in (0,\infty)$ & $\infty$\\
			\cline{1-4}
			$0$ 			   & $0$ & $a$ 		& $\infty$\\
			$b \in (0,\infty)$ & $b$ & $(a^{p}+b^{p})^{1/p}$		& $\infty$\\
			$\infty$ 		   & $\infty$ & $\infty$ & $\infty$
		\end{tabular}
		\hspace*{10ex}
		\begin{tabular}{c|ccc}
			\textnormal{$a \phsum{p} b$} & $0$ & $a \in (0,\infty)$ & $\infty$\\
			\cline{1-4}
			$0$ 		 	   & $0$ 		& $0$ 	   & $0$\\
			$b \in (0,\infty)$ & $0$ 		& $(a^{-p}+b^{-p})^{-1/p}$	   & $b$\\
			$0$ 		   & $0$ 	& $a$ & $\infty$
		\end{tabular}
	\end{equation*}
    Where $p \in (0,\infty].$
\end{definition}

\begin{lemma}[Additive Collapse]
    The following are the limits of $\psum{p}$ and $\phsum{p}$ for $p \longrightarrow \infty$:
    \begin{equation*}
		\begin{tabular}{c|ccc}
			$a \psum{\infty} b$ & $0$ & $a \in (0,\infty)$ & $\infty$\\
			\cline{1-4}
			$0$ 			   & $0$ & $a$ 		& $\infty$\\
			$b \in (0,\infty)$ & $b$ & $\max{}(a,b)$		& $\infty$\\
			$\infty$ 		   & $\infty$ & $\infty$ & $\infty$
		\end{tabular}
		\hspace*{10ex}
		\begin{tabular}{c|ccc}
			\textnormal{$a \phsum{\infty} b$} & $0$ & $a \in (0,\infty)$ & $\infty$\\
			\cline{1-4}
			$0$ 		 	   & $0$ 		& $0$ 	   & $0$\\
			$b \in (0,\infty)$ & $0$ 		& $\min{}(a,b)$	   & $b$\\
			$\infty$ 		   & $0$ 	& $a$ & $\infty$
		\end{tabular}
	\end{equation*}
\end{lemma}

\begin{definition}[Multiplication]
\label{Multiplication}
    On $[0,\infty]$, \emph{conjunctive multiplication} and \emph{disjunctive multiplication} are, respectively, the following operations:
    \begin{equation*}
		\begin{tabular}{c|ccc}
			$a \conmul{} b$ & $0$ & $a \in (0,\infty)$ & $\infty$\\
			\cline{1-4}
			$0$ 			   & $0$ & $0$ 		& $0$\\
			$b \in (0,\infty)$ & $0$ & $ab$		& $\infty$\\
			$\infty$ 		   & $0$ & $\infty$ & $\infty$
		\end{tabular}
		\hspace*{10ex}
		\begin{tabular}{c|ccc}
			\textnormal{$a \dismul{} b$} & $0$ & $a \in (0,\infty)$ & $\infty$\\
			\cline{1-4}
			$0$ 		 	   & $0$ 		& $0$ 	   & $\infty$\\
			$b \in (0,\infty)$ & $0$ 		& $ab$	   & $\infty$\\
			$\infty$ 		   & $\infty$ 	& $\infty$ & $\infty$
		\end{tabular}
	\end{equation*}
\end{definition}

Notice $\conmul{}$ and $\dismul{}$ differ only when $a$ is $0$ and $b$ is $\infty$, or \textit{vice versa}. Often we write $ab$ instead of $a \conmul{} b$.

\begin{definition}[Division]
\label{Division}
    On $[0,\infty]$, \emph{division} is:
    \begin{equation*}
		\begin{tabular}{c|ccc}
			$a \ediv{} b$ & $0$ & $a \in (0,\infty)$ & $\infty$\\
			\cline{1-4}
			$0$ 			   & $\infty$ & $0$ 		& $0$\\
			$b \in (0,\infty)$ & $\infty$ & $b/a$		& $0$\\
			$\infty$ 		   & $\infty$ & $\infty$ & $\infty$
		\end{tabular}
	\end{equation*}
\end{definition}

\begin{definition}[Duality Operator]
\label{dual}
    Let $a \in [0,\infty]$. Then the \emph{dual} of $a$ is
    \[  \edual{a} =
    \begin{cases}
    1/a  & a \in (0,\infty)  \\
    \infty & a = 0 \\
    0 & a = \infty \\
   \end{cases}
    \]
\end{definition}

Notice $a^{-1} = a \ediv{} 1$,  $a \ediv{} b = a^{-1} \dismul{} b$ and 
$a \dismul{} b = (a^{-1} \conmul{} b^{-1})^{-1}$. We sometimes slightly abuse the notation to write $a/b$ instead of $b \ediv a$ and $1/a$ instead of $\edual{a}$.

\begin{comment}
\subsection{Measure Spaces and $p$-means}\label{p-mean}

%\begin{definition}[Measurable Spaces and Functions]
%    Let $S, T$ be sets and $\sigmal{S}, \sigmal{T}$ be 
%    $\sigma$-algebras. The pairs $(S, \sigmal{S})$ and $(T, \sigmal{T})$
%     are \emph{measurable spaces}.\\ The function $f : S \rightarrow T$ is
%      \emph{measurable} over $S$ with values in $T$ if and only if for every 
%      $E \in \sigmal{T}$ the preimage of $E$ under $f$ is in $\sigmal{S}$, 
%      that is, for all $E \in 
%      \sigmal{T}$
%    \begin{equation}
%       f^{-1}(E) = \{x \in S_{1} \, | \,f(x) \in E \} \in \sigmal{S}.
%    \end{equation}
%\end{definition}

\begin{definition}[Measure Space]
    Let $S$ be a set and $\sigmal{S}$ be a $\sigma$-algebra over $S$. 
    A \emph{measure} on $(S,\sigmal{S})$ is a function 
    $\mu : \sigmal{S} \rightarrow [0,\infty]$ such that (1) $\mu (\varnothing) 
    = 0$ and (2) if $\{ A_i : i \in I \}$ is a countable collection of pairwise 
    disjoint sets in $\sigmal{S}$ then
    \begin{equation}
        \mu \left( \bigcup_{i \in I} A_i \right) = \sum_{i \in I} \mu (A_i).
    \end{equation}
    The triple $(S, \sigmal{S}, \mu)$ is called a \emph{measure space}, and a 
    \emph{probability space} when $\mu(S)=1$, in which case $\mu$ is often 
    denoted as $\mathbb{P}$.
\end{definition}

%\begin{definition}[Random Variable]
%    Let $S,T$ be measurable spaces. A \emph{random variable} over $S$ 
%    with values in $T$ is a measurable function 
%    $X : S \rightarrow T$.
%\end{definition}

We give the following definitions for positive functions only, since this is 
the integrals we use below.

\begin{definition}[Simple Functions and Lebesgue Integral]
    Let $(S, \sigmal{S})$ be a measurable space, $I$ be a finite index set, 
    $a_i \in \mathbb{R}$ for each $i \in I$ and $\{ A_i : i \in I\}$ a 
    collection of sets in $\sigmal{S}$. A \emph{simple function} on $S$ is one
     that can be written as a finite linear combination of indicator functions 
     of measurable subsets of $S$, i.e. one of the form 
     $f = \sum_{i \in I}a_{i}\emph{1}_{A_i}$.
    If $\emph{S} = (S, \sigmal{S}, \mu)$ is a measure space then:
    \begin{enumerate}
        \item If $f = \sum_{i \in I}a_{i}\emph{1}_{A_i}$ is a nonnegative 
        simple function, the \emph{Lebesgue integral} of $f$ is
        \begin{equation}
            \int_{\emph{S}} f = \int_{S} f(s) \, \mu(\de s) = 
            \sum_{i \in I} a_i \conmul \mu(A_i).
        \end{equation}
        \item If $f : S \rightarrow [0, \infty]$ is a measurable function, the \emph{Lebesgue integral} of $f$ is
        \begin{equation}
            \int_{\emph{S}} f = \int_{S} f(s) \, \mu(\de s) = 
            \sup\left\{ \int_{S} g : g \text{ is simple and } g \leq f \right\}.
        \end{equation}
    \end{enumerate}
\end{definition}

The following definitions relate specifically to the new quantifier semantics. 
They are what are classically known as generalized weighted means 
\cite{mitrinovic1970analytic}, though the geometric mean, much like multiplication 
above, bifurcates into a conjunctive and a disjunctive version.
\\\\
Throughout the following, fix a probability space 
$\emph{S} = (S, \sigmal{S}, \mathbb{P})$.


\begin{definition}[$p$-Means]
\label{pmean}
    Let $f : S \rightarrow \PEreal$ be a measurable function. For $p \in (0, \infty)$, the \emph{(generalized weighted) $p$-mean} of $f$ is
    \begin{equation}
        %\LMS{f}{p}{S} := \left(\int_{S} f(s)^p\, \mathbb{P}(\de s)(s)\right)^{1/p}
        \LMS{f}{p}{\emph{S}} := \left(\int_{\emph{S}} f^{\,p}\right)^{1/p} = 
        \left(\int_{S} f(s)^{\,p}\, \mathbb{P}(\de s)\right)^{1/p}
    \end{equation}
    where we extended the functions $(-)^p$ as follows
    \begin{equation}
        \infty^{p} =
        \begin{cases}
            1  & p = 0  \\
            \infty & p > 0
        \end{cases}
        \hspace*{10ex}
        0^{p} = 0.
    \end{equation}
    Dually, the \emph{(generalized weighted) harmonic $p$-mean} of $f$ is
    \begin{equation}
        \LMS{f}{-p}{\emph{S}} := 
        \left(\LMS{f^{-1}}{p}{\emph{S}}\right)^{-1} = 
        \left(\int_{\emph{S}} f^{\,-p}\right)^{-1/p} =
        \left(\int_{S} f(s)^{\,-p}\, \mathbb{P}(\de s)\right)^{-1/p}.
    \end{equation}
\end{definition}

When $\emph{S}$ can be inferred from the context (for example, when $f$ is a random 
variable), we write $\LM{f}{p}$.

The definition of $p$-means can be extended to $p=0$ and $p=\infty$ by taking limits \cite{capucci}. First we have

\begin{lemma}
\label{limitinfty}
    As $p \longrightarrow +\infty$,
    \begin{equation}
        \LM{f}{+p} \longrightarrow \esup{f} =: \LM{f}{+\infty},
        \qquad
        \LM{f}{-p} \longrightarrow \einf{f} =: \LM{f}{-\infty}.
    \end{equation}
\end{lemma}

These quantities are so defined:

\begin{definition}[Essential Extrema]
    Let $(S, \sigmal{S}, \mu)$ be a measure space and $f : S \rightarrow \PEreal$ a measurable function.
    \begin{enumerate}
        \item Let $U = \left\{ a \in \PEreal : \mu(\{ x \in X : a < f(x)\}) = 0\right\}$ and $\inf(U)$ be the infimum of U. The \emph{essential supremum} of $f$
        is
        \begin{equation}
            \esup{f} = \inf U
        \end{equation}
        recalling that $\inf \varnothing = \infty$.
        \item The \emph{essential infimum} of $f$ is
        \begin{equation}
            \einf{f} = - \,\esup{- f}
        \end{equation}
    \end{enumerate}
\end{definition}

On the other end of the spectrum, we have:

\begin{lemma}
\label{limitzero}
    As $p \longrightarrow 0$, both $\LM{f}{+p}$ and $\LM{f}{-p}$ converge to a limit, thus defining \emph{disjunctive} and \emph{conjunctive geometric means}:
    \begin{equation}
        \LM{f}{+p} \longrightarrow: \LM{f}{+0},
        \qquad
        \LM{f}{-p} \longrightarrow: \LM{f}{-0}.
    \end{equation}
\end{lemma}

For bounded functions, these quantities coincide with the classical (weighted) geometric mean:

\begin{definition}[Geometric Mean]
    Let $f : S \rightarrow [0,\infty)$ be a measurable function and $(S,\sigmal{S}, \mathbb{P})$ a measure space. The \emph{geometric mean} of $f$ is
    \begin{equation}
        GM[f] = \exp \left(\int_{S} \ln f(s)\, \mathbb{P}(\de s)\right)
    \end{equation}
\end{definition}

For unbounded functions, conjunctive and disjunctive geometric means may differ in the same way as $\conmul{}$ and $\dismul{}$, namely in the way they handle $0$ and $\infty$. See \cite{capucci2024quantifiers} for clarifications.
\end{comment}

\subsection{Syntax}

\citeauthor{slusarz2023logic} propose a common syntax for all DLs \citep{slusarz2023logic}. Here we adapt a subset of it.  For simplicity we use the same symbols of \cref{Preliminaries} for our formulas. 

\cref{fig:syntax} defines the syntax of \OL{}. Types are given by Boolean or  Extended Positive Real Numbers ($(0,\infty]$). Formulas are freely generated from a set of atomic propositions over logical connectives that resemble those of linear logic \citep{Wadler1993, agliano2025algebraic},  excluding that the additive connectives are parametrised by a positive extended real $p$.\\

\begin{figure}[H]
\begin{subfigure}[t]{0.4\textwidth}
	\begin{grammar}
		<type> ::=  
        \BoolType | \ERealType 

        
	\end{grammar}
\end{subfigure}
\hfill
\begin{subfigure}[t]{1\textwidth}
	\begin{grammar}
		<exprEPReal> $p$ ::=  
        $p \in (0, \infty]$ 
	\end{grammar}
\end{subfigure}
\begin{subfigure}[t]{1\textwidth}
	\begin{grammar}
		<exprBool> $\ni \phi_{0},\phi_{1}$ ::=  
        $\one$ | $\top$ | $\bot$ | $\phi_{0} \ediv{} \phi_{1}$ | $\phi_{0} \conmul{} \phi_{1}$ | $\phi_{0} \psum{\elEReal} \phi_{1}$ | $\phi_{0} \phsum{\elEReal} \phi_{1}$ 
	\end{grammar}
\end{subfigure}
\hfill
\setlength{\belowcaptionskip}{-20pt} 
	\caption{Types and expressions of \OL{}.\\
	}
	\label{fig:syntax}
\end{figure}

\subsection{Semantics}


\subsection{Sequent Calculus}
\subsection{Formalization}

